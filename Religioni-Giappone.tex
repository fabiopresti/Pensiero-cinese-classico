\documentclass[10pt,a4paper]{report}
\usepackage[top=1in, bottom=1in, left=1.2in, right=1.2in]{geometry} 
\usepackage[utf8]{inputenc}
\usepackage[T1]{fontenc}
\usepackage[italian]{babel}
\usepackage{amsmath}
\usepackage{amsfonts}
\usepackage{amssymb}
\usepackage{makeidx}
\usepackage{placeins}
\usepackage{graphicx}
\author{Fabio Prestipino}
\title{Religioni e filosofie del Giappone}
\begin{document}
	\maketitle
\section*{Periodizzazione}
\begin{itemize}
	\item Periodo Jomon 10000 a.C.-300 a.C. (mitica fondazione dell'impero da parte di Jinmu)
	\item Periodo Yayoi 300 a.C.-250 a.C. (inizio della coltivazione del riso)
	\item Periodo Kofun 250 a.C.-538 d.C. (graduale centralizzazione del potere attorno ai clan)
	\item Periodo Asuka 538 d.C.-710 d.C. (primo periodo buddhista)
	\item Periodo Nara 710 d.C.-794 d.C. (fondazione della capitale Nara)
	\item Periodo Heian 794 d.C.-1185 d.C. (capitale Kyoto)
	\item Periodo Kamakura 1185 d.C.-1333 d.C.(fioritura shinto buddhista)\\
	...\\
	\item Periodo Edo 1603-1868
	\item Periodo Meiji 1868-1912 (shinto utile per la nazione, 13 sette)
\end{itemize}
La vera fioritura del Giappone si ha una vera fioritura durante il dialogo con la gloriosa dinastia Tang, di cui sono ammaliati. Nonostante questo il taoismo di epoca Tang non penetra in Giappone perché vedono nel Taoismo un'analogo dello Shinto, e dunque non se ne interessano. I Giapponesi cercano e assorbono ciò che è diverso da loro, che gli manca, ed è per questo che il Buddhismo sarà rapidamente assimilato.
\chapter{Shinto}
La parola Shinto ("la via dei kami"), usata per indicare la religione giapponese tradizionale, è stata assegnata successivamente per distinguerla dal buddhismo, non vie era un termine inizialmente per riferirsi all'insieme di credenze tradizionali. Questa religione è praticata solamente in Giappone e non ha fondatore ma si è sviluppata naturalmente. \'E legata al culto dei kami, una varietà di divinità naturali, per i quali si fanno feste, discipline ascetiche e attività sociali. Non esistendo testi sacri assume caratteristiche mutevoli a seconda dei luoghi e dei periodi. Ha modellato la cultura giapponese non per mezzo di una filosofia ma con il suo codice di valori fondamentali e le sue generali modalità di pensiero. Oggi si possono distinguere tre tipologie di Shinto molto legate fra loro e spesso mischiate:
\begin{itemize}
	\item Shinto dei santuari: risalente all'inizio della storia del Giappone, è la componente principale di questa religione ed ha avuto un ruolo di unificazione nella società attorno a grandi santuari e mediante feste e tradizioni. La massima divinità è Amaterasu (dea della luce) affiancata da divinità naturali ed antenati. La preghiera ha anche funzione propiziatoria per un pacifico sviluppo del Paese, rappresentato dall'imperatore.  
	\item Shinto imperiale: basato sui riti relativi alla famiglia imperiale
	\item Shinto popolare: diffusa nelle fasce basse della società non ha un vero sistema di pensiero o un'organizzazione ecclesiastica. \'E basata su tre fonti: la sopravvivenza di antiche tradizioni magiche come la divinazione, riti di ascenzione e purificazione e culto delle divinità agresti e domestiche, influenze da altre religioni come buddhismo, taoismo e cattolicesimo.  
	\item Shinto delle sette: incentrato su 13 gruppi formatisi nel XIX secolo, ognuna con un fondatore. Non hanno templi ma si organizzano in strutture simili alle nostre chiese. Nel dopoguerra vi sono state molte scissioni ed altre sette sono nate e, anche se non si professano strettamente shintoiste, sono legate a questo terreno culturale.
\end{itemize}
\section{Panoramica storica}
\subsection{Lo shinto antico}
Per Shinto antico si intende quello precedente agli influssi confuciani e buddhisti, terminato nel VII secolo d.C. La mitologia giapponese antica è simile a quella dei gruppi etnici limitrofi, ad evidenziare il legame antico fra queste culture. La parola kami era usata anticamente come aggettivo per indicare qualcosa di misterioso, soprannaturale e sacro (con accezione di spaventoso). A partire dai semplici oggetti del culto fino alle grandi divinità o agli antenati sono detti kami. I kami possono essere divinità naturali, antropomorfe (antenati, eroi o grandi personaggi) o astratte. Gli antenati si credeva diventassero spiriti dopo un periodo di purificazione e tornassero dove avevano vissuto una volta l'anno. Il kami più importante era la divinità del clan che proteggeva l'unità sociale basilare del periodo: il clan. Questo non era sempre un antenato ma poteva essere anche legato ai mezzi di sussistenza del clan o alla realtà geografica o politica. In origine non c'erano santuari e i riti venivano svolti in luoghi naturali considerati sacri; in seguito costruirono capanne in corrispondenza dei luoghi sacri per proteggere loro stessi e l'altare dalle intemperie che gradualmente divennero i santuari attuali. L'idea del kami si è evoluta di pari passo, partendo dall'idea di una divinità che saltuariamente fa visita ai fedeli ad una divinità sempre presente nel santuario. Visto che l'agricoltura è stata il principale mezzo di sussistenza fino a tempi recenti i riti centrali erano relativi alla vita agreste. Per determinare la volontà e gli umori dei kami si praticava un tipo di divinazione in cui si bruciava la scapola di un cervo e se ne esaminavano le crepe. Ma si usavano anche i sogni  e si consultavano oracoli in possessione spiritica.\\
Anticamente coesistevano due cosmologie: una secondo cui il mondo è strutturato verticalmente su tre livelli cielo, terra di mezzo e inferi un'altra secondo cui vi sono due mondi orizzontali, uno che è quello comune ed un altro eterno, collocato ai limiti del mondo fenomenico. La prima è tipica dell'Asia del nord e la seconda del sud, quest'ultima ha una molto minore influenza sulla cultura giapponese. Un fatto interessante è che in entrambe i casi i mondi sovrannaturali sono continuazioni e riflessi del mondo reale e non hanno maggior valore rispetto al mondo degli uomini. La mite vita campestre porta ad accettare la realtà così com'è.
\subsection{Gli influssi confuciani}
Lo Shinto antico si sviluppa sotto la spinta cinese, in particolare la moralità confuciana fa ingresso in Giappone verso il V secolo e comporta una ricerca di un fondamento morale nei kami. Inoltre cominciò un percorso di unificazione e ordinazione dei miti all'interno della cornice dei miti della famiglia imperiale. La nazione unificata portò ad istituzionalizzare le divinità Shinto principali e Amaterasu divenne inizialmente la divinità tutelare del clan dell'imperatore e poi la dea protettrice della nazione. Lo Shinto cominciò ad essere gestito dal potere centrale. Nacquero nuovi riti molto simili a quelli cinesi ma reinterpretati sulla base di riti già esistenti in Giappone. 
\subsection{La fusione fra Buddhismo e Shinto}
Se inizialmente il Buddha fu preso come "kami del paese vicino" con il passare del tempo la profondità del buddhismo e la specificità della sua iconografia affermarono la sua unicità e conquistarono i cuori della gente. La fede nella terra pura di Amida (Amidismo) fu la prima e più semplice versione di buddhismo a diffondersi in Giappone. Inizialmente i kami vennero percepiti come protettori del Buddha e la venerazione di questi nei templi buddhisti accompagnava quella per il Buddha; in una seconda fasi si diffuse nel clero buddhista l'idea dei 10 regni dell'essere (diversi stati di esistenza a partire dagli inferi fino al Buddha) e si cominciò a pensare che i kami occupassero le posizioni più elevate nel regno dell'illusione; la terza fase vede i kami come incarnazioni dei bodhisattva o come manifestazioni fenomeniche dei Buddha. L'iconografia buddhista è spesso mista a quella Shinto.\\
Nel periodo Kamakura fiorì lo Shinto buddhista attorno alle teorie che i monaci buddhisti avevano riguardo i kami. Un esempio è lo Shinto Tendai che percepiva il Buddha Sakyamuni come la realtà ultima che sta dietro tutti i fenomeni, kami inclusi. Questo può far intuire quanto i rapporti fra le due religioni fossero intricati. Questa situazione dura fino alla fine del periodo Edo ma già dal periodo Kamakura si sviluppava contemporaneamente una corrente autonoma e antibuddhista di Shintoismo. Lo Shinto Watarai ad esempio, una corrente antibuddhista, sosteneva che la forma originaria, anche precedente ai Buddha, fosse il caos e successivamente il non essere. Lo Shinto Yoshida invece identificava una divinità fondamentale come creatrice di tutto l'universo e che l'anima umana può diventare sede di un kami e dunque bisogna mantenere un atteggiamento mistico per poterla accogliere. Lo Shinto Yoshida era la principale religione svincolata dal buddhismo.
\subsection{La restaurazione dello Shinto di epoca Edo}
All'inizio del periodo Edo lo Shinto vede una nuova fioritura grazie ad una spinta teorica offerta dalla sintesi con il confucianesimo. Questa variante di Shinto si diffonde nel ceto guerriero ed è vista come una visione del mondo adatta a governare.\\
Durante il periodo Edo nasce uno sforzo di recuperare l'ethos del Giappone antico "restaurando" lo Shinto mediante la ricerca letteraria. Si rifiutano dunque gli influssi confuciani e buddhisti e si ricerca lo Shinto originario, rilanciando la figura di Amaterasu come "antenata imperiale" ed attribuendo alla volontà dei kami l'evoluzione di ogni cosa nel mondo. In questo periodo si sviluppa una teologia Shinto influenzata dal cattolicesimo. 
\subsection{Lo Shinto popolare del periodo Meiji}
Con l'evolversi della società i culti popolari si incentrano sulle unità territoriali (proprietà terriere) e sui villaggi più che sui clan. Con il conservatorismo del periodo Edo si riprendono tradizioni Shinto cadute in disuso, come la periodica ricostruzione del Grande santuario di Ise (tutt'oggi visto come il principale santuario Shinto), il sumo ed il tiro con l'arco. Il governo Meiji addirittura emanò una direttiva politica per separare Buddhismo e Shinto allo scopo di rendere la restaurazione dello Shinto il fondamento spirituale della neonata nazione giapponese occidentalizzata. Lo Shinto era infatti visto come mezzo di unificazione, specialmente per integrare le zone rurali, ma questo venne epurato da alcune pratiche come la divinazione e le preghiere pr invocare guarigioni, che vennero invece proibite. La forma che prese lo Shinto in questo periodo è simile a quella attuale.\\
Nel dopoguerra le forze di occupazione emanarono direttive che spezzarono il rapporto fra Shinto e Stato, relegando lo Shinto a comunità periferiche. \\
L'amministrazione Meiji accettò ufficialmente 13 delle innumerevoli sette Shinto formatesi nel tempo. Queste si sono sviluppate grazie ad un periodo di crisi sociale nella transizione tra periodo Edo e Meiji e la conseguente crisi del buddhismo, religione ufficiale che dopo la perdita del sostegno dello Shogun aveva perso d'attrattiva. A partire dal dopoguerra le sette si sono frammentate ed il quadro attuale è molto complesso. 
\section{I testi, mitologia e credenze}
Nonostante non vi sia un fondatore o un canone vi sono alcuni testi antichi e rispettati. Fra questi ricordiamo il Kojiki ed il Nionshoki, testi sacri in forma di annali che trattano la storia del Mondo e del Giappone dalle origini, a partire dalla nascita degli dei, la fondazione della nazione giapponese e la sua unificazione sotto l'impero, fondatosi su volontà divina e rispetto dei kami. Quando Cielo e Terra vennero divisi i kami apparvero spontaneamente ed in particolare due di loro, Izanagi e Izanami,  generarono il Giappone, la sua natura e i capostipiti dei clan, oltre che altri kami come Amaterasu, e Susanoo (sovrano degli inferi).  I miti narrano fondamentalmente di Amaterasu e di come i suoi discendenti siano giunti sulla terra per unificare il Giappone. Jimmu, discendente di Amaterasu, è riconosciuto come il leggendario primo imepratore del Giappone. A partire dal dopoguerra l'insegnamento di questa cultura popolare è praticamente scomparso dalle scuole.\\
Lo Shinto ruota attorno al concetto di kami: la loro essenza non si può spiegare a parole, sono misteriosi ma i credenti possono conoscere la loro volontà grazie alla loro fede. Il politeismo dei kami è legato alla visione dell'essere giapponese, ma è contemporaneamente possibile individuare la credenza in una fonte originaria di tutte le cose, e dunque anche dei kami, che però è sempre stata marginale, non c'è un celato monismo.\\
Il presupposto dello Shintoismo è che gli uomini sono figli dei kami e l'uomo riceve la vita dai kami, che dunque è sacra. I kami sono dunque da onorare e dalla sacralità del dono dei kami discende un concetto di dignità personale di tutti gli uomini, che sta a fondamento dei moderni movimento Shintoisti per la pace. Non è rintracciabile qualcosa di simile al peccato originale e l'uomo possiede dentro di sè l'essenza stessa dei kami che però si mostra solo quando l'uomo è puro. La purezza d'animo, l'onestà, la sincerità sono le virtù umane cardine dello Shintoismo. L'etica Shinto consiste nello svolgere al meglio il proprio lavoro, ed adempiere alle virtù morali individuali; le virtù discendono dal magokoro ("mente pura"), la purezza è prerequisito per essere in contatto con i kami. Nonostante vi sia una concezione dell'individuo come essere particolare, dotato di sue proprie caratteristiche, questo non è percepito come totalmente indipendente ma inscritto nella sua tradizione, che lo collega verticalmente ad antenati e discendenti. Orizzontalmente invece l'uomo è inscritto nelle relazioni con i gruppi sociali con cui ha a che fare, come il nodo di una catena. Se però si parla dell'individualità dell'uomo, il concetto fondamentale è quello del "presente al centro": il tempo è visto come un flusso eterno con il suo centro nel presente, luogo di incontro tra individuo e storia eterna, questo porta a valorizzare l'istante che è da vivere nel modo più pieno e ricco di senso possibile. 
\section{L'arte}
I Giapponesi eccellono nell'esprimere la loro esperienza religiosa in forma estetica ed emozionale più che con costruzioni logiche e filosofiche, la natura che pervade lo spirito del giapponese lo porta a vederla come opera del kami. I santuari sono essi stessi opere d'arte: immersi nella natura, in un clima calmo ma solenne, ispirano un senso d'armonia fra natura e architettura che rispecchia quello fra uomo e kami, che vivono in simbiosi. Emblematici gli antichi santuari di Ise e Izumo, che essendo precedenti al IX secolo non risentono ancora dell'influenza delle complesse strutture cinesi. Un elemento tipico è l'entrata sacra del santuario: il torii, di origine sconosciuta, davanti al quale si trova spesso una coppia di cani o leoni contro gli influssi maligni.\\
Le immagini dei kami si introducono nei santuari a seguito dell'interazione con il buddhismo, ed erano usate per illustrare i misteri e la storia del santuario. Le sculture dei kami invece inizialmente erano nascoste nel sacrario interno e non venivano usate per il culto diretto, e si diffusero pubblicamente solo in risposta al buddhismo, le due forme artistiche si fusero gradualmente. Anche l'interazione della corte con lo Shinto ebbe forte influenza sull'arte tanto che talvolta i kami sono rappresentati come uomini e donne dell'aristocrazia.\\
La musica shintoista non inneggia alla divinità ma placa i kami o rallegra l'animo degli uomini. Ciò probabilmente deriva dall'antica credenza di dover creare un'atmosfera piacevole durante le feste per placare l'ira dei kami, causa di disastri naturali. La musica divenne solenne, fatta di preghiere e inni, a seguito dell'interazione con il confucianesimo. La musica e danza gagaku è tipica shintoista e autenticamente giapponese.  
\subsection{Lo Shinto oggi}
Nella storia lo Shinto ha svolto due principali funzioni sociali: fornire una soluzione ai fondamentali problemi di senso e della vita individuale in generale, sviluppare un senso di comunità a partire dalla famiglia, il villaggio fino alla nazione intera. Questa seconda funzione è palese nello Shinto dei santuari che nella storia si è spesso legato al potere politico, che doveva essere moralmente virtuoso anche mediante la religiosità. Anche in periodi di crisi politica i potenti tutelavano i santuari e vi pregavano. I Meiji sfruttarono a loro favore questa funzione di collante sociale e adottarono lo Shinto come fondamento ideologico della nazione. Nel dopoguerra, in nome della separazione fra stato e religione, si smise di finanziare i culti e la Costituzione tutt'ora vigente conferma questa linea, ancora molto discussa (all'interno di un dibattito più ampio sulla compatibilità fra tradizioni culturali e democrazia). Nonostante sia in crisi, lo Shinto è talmente legato alla cultura giapponese da non poter essere ignorato nel trattare l'identità culturale di questo paese. 
\section{Shinto approfondimento}
Religione vitalista, celebra ogni aspetto della realtà esperita dall'uomo ed ha una forte paura della morte, vista come atto massimamente impuro. Visione totalmente opposta a quella buddhista per cui la morte è liberazione, lo Shinto è tradizionalmente riuscito a dialogare con il buddhismo perché mentre quest'ultimo si occupa dell'assoluto, lo Shinto ha a che fare con l'esperienza umana, si dividono i compiti. Nonostante il Giappone sia fortemente areligioso, il giapponese vive (più o meno consapevolmente) in società da shintoista e dialoga con la morte e con l'arte da buddhista.\\
La natura ha un ruolo importante nello Shinto, lo si evince anche dai luoghi religiosi. La concezione cosmologica è di tipo generativo: dai kami discendono gli umani ma anche la flora, la fauna e la natura inanimata e l'uomo è dunque segretamente imparentato con tutto ciò che lo circonda. Un tratto interessante è il disinteresse shintoista per l'etica, legata all'orrore verso alcuni comportamenti ma basata su fatti emotivi più che di riflessione filosofica. Vi è inoltre una forte commistione fra etica ed estetica. Storicamente ciò può essere dovuto alla precoce influenza di filosofie già molto complesse e con una filosofia morale ben strutturata. Il "non uccidere" buddhista penetrerà tanto in profondo in Giappone da rendere la maggior parte della popolazione pescatariana.  
\subsection{Origini}
Sappiamo di una brusca rivoluzione culturale nel 300 a.C. dal periodo Jomon allo Yayoi, si ipotizza vi sia un sostrato ancestrale di credenze in asia orientale in cui shinto e taoismo erano mischiate, questa religione nel continente prese una direzione più filosofica andando verso il taoismo cinese mentre in Giappone una mentalità artistica ed intuitiva favorì lo sviluppo dello shinto (taoismo filosofico contro mitologia shinto). Alcuni considerano lo shinto una "versione travestita del taoismo". Una prima forma di Shinto in periodo Yayoi risponde al contatto con la molteplicità della natura mediante un politeismo sfrenato (milioni di kami). Il concetto di kami non è ben definito: un'oggetto (ade sempio il monte Fuji) è simbolo, sede o dimora di un kami? Le fonti storiche in Giappone cominciano ad essere cospicue solo a partire dal VII secolo d.C. e le prime sporadiche risalgono al III secolo d.C.
\subsection{Lo Shinto dei clan}
Durante il III secolo si passa dall'era Yayoi a quella kofun (tombe a tumulo), che dura fino al 645 d.C. Aumenta la complessità delle società che si dividono i clan legati al territorio e che si svilupperanno in grossi paesi. In questo periodo si sviluppa un forte culto degli antenati ed in particolare dell'uji-gami, il kami dell'antenato fondatore del clan. I ritrovamenti di ciò che veniva lasciato ai morti nelle tombe rispecchia perfettamente l'evoluzione di una società più centralizzata e guerriera. 
\subsection{Lo Shinto nel regno di Yamato}
Nel VI secolo si conclude un lungo periodo di selezione fra i clan emersi che vede il prevalere degli Yamato, una dinastia già potente e ben formata. Dta la mancanza della scrittura questa dinastia scrisse una mitologia/storia utile all'affermarsi del loro dominio che legava gli antenati Yamato ai kami. In questo contesto si diffonde la scrittura e i contatti con il continente si infittiscono ed il panorama culturale e religioso muta profondamente. Il Giappone si fa discepolo della dinastia cinese Tang. La mitologia Shinto che si solidifica mediante la scrittura in questo periodo non avverte più gli antichi legami con l'Asia continentale ed è interamente centrata sul Giappone, si parla di "nippogonia", questo tratto graverà sulla cultura giapponese che tende a ritenersi un popolo speciale, eletto. Ancora una volta notiamo l'assenza della creazione nella mitologia giapponese, sostituita alla generazione. Conseguenza principale di questo atteggiamento è una mancanza di nette distinzioni tra sfera umana e divina, che sono ontologicamente più amalgamate. La mitologia cela un indizio di un importante avvenimento preistorico: l'accordo di pace tra gli Yamato e gli Izumo, ai primi toccarono le divinità celesti ed ai secondi quelle terrestri. Gli dei primordiali, sono Izanami e Izanagi ("il Maschio che invita", "la Femmina che invita"). Dopo alcuni tentativi Izanami ed Izanagi riescono ad avere un rapporto armonioso generando le isole del Giappone, inizialmente Izanami comincia il rapporto e questo fallisce in un aborto, gli dei allora danno il responso di rifare tutto al contrario (cioè far cominciare l'uomo), in questo modo le generazioni procedono spedite. Questo particolare potrebbe rispecchiare il passaggio sofferto da una società prevalentemente matriarcale (di cui si hanno varie testimonianze) ad una patriarcale. Fatto interessante è che Izanami muore dopo aver partorito il dio del fuoco e che Izanagi fa un viaggio negli inferi che trova il corpo di Izanami in putrfazione e fugge (echi del ito di Orfeo ed Euridice). Uscito dagli inferi Izanagi è preso dalla smania di purificarsi. Momento centrale nella mitologia Giappoense, da questa purificazione discende la nascita degli dei principali fra cui Amaterasu e Susanoo che garantiranno l'ordine del mondo degli umani. 
\subsection{Shinto e fedi continentali}
Nel VI secolo cominciano gli influssi buddhisti dalla Corea in Giappone. Il buddhismo era inizialmente concepito come una forma superiore di magia e non si avrà una profonda comprensione del buddhismo fino all'eppoca di Nara (700 d.C. circa.). La ricezione è inizialmente legata alla politica in quanto accettare il buddhismo avrebbe fortificato il potere centrale ed indebolito gli attuali privilegi dei clan. A partire dal sovrano reggente Shotoku in poi il buddhismo si sarebbe imposto fino a diventare la religione principale. Come già osservato Shinto e Buddhismo stanno culturalmente agli antipodi, a differenza di confucianesimo e taoismo (in particolare le frange più mistiche). La possibilità di facili interazioni fra religioni è favorita dalla concezione stessa dell'esperienza religiosa, non esclusiva, profondamente diversa da quella occidentale. Il kami Hachiman, inizialmente un dio della guerra ancestrale, che a seguito di vicende al limite fra realtà e mito, venne nominato Grande Bodhisattva, uno dei pochi kami di cui si è diffusa l'iconografia. L'esempio di Hachiman dimostra come sia possibile il dialogo, anche locale, di buddhismo e shinto. L'inferiorità concettuale dello shinto rispetto al buddhismo porta ad un complesso d'inferiorità del primo ch porta a vedere i kami come esseri subordinati al karma; i kami erano manifestazioni locali degli archetipi degli Spiriti Magni buddhisti.\\
Nel XI secolo Kukai e Saicho importano il buddhismo Shingon e Tendai, di spiccata tendenza esoterica. Dalla dinastia Kamakura in poi il buddhismo fu il principale portatore di cultura. Un fatto interessante è che una delle poche idee la cui importazione fu totalmente rifiutata è quella del concetto di mandato celeste cinese, che in Giappone è declinato come mandato nei confronti di una dinastia. Il tempio di Ise è un esempio di rigetto del buddhismo da parte dello Shinto, il Watarai Shinto, sostenendo addirittura che i Kami hanno priorità sugli archetipi buddhisti, che sono loro emanazione. Questo shintoismo dai tratti nazionalisti rimarcò la specialità del popolo giapponese con effetti a lungo termine devastanti.\\
Con il dialogo con cristianesimo e protestantesimo avvenuto tra il 1500 ed il 1600 l'interesse per il buddhismo si affievolisce e shinto (in particolare Watarai) e neo-confucianesimo tornano in auge, producendo un credo illuminista, anti cinese ed anti buddhista.\\
Un evento che segnò la storia del Giappone è l'arrivo delle navi nere del commodoro Mattew Perry nel 1853, che sancì la presa di coscienza dell'inferiorità militare giapponese nei confronti dell'america e portò ad una graduale apertura del Giappone verso il resto del mondo. In questo contesto fiorì un "fondamentalismo shinto" caratterizzata da un'organizzazione teocratica della società e da un'intolleranza verso le altre religioni (con particolare sfortuna per i buddhisti). Lo Shinto venne istituzionalizzato, favorendo lo Shinto dei sacrari, che potevano essere controllati dal potere centrale, il rapporto con le altre religioni si distese.\\
Dopo la seconda guerra mondiale il Giappone fu occupato dagli americani, uno dei primi provvedimenti fu quello, da parte del generale McArthur, di eliminare i fondi statali per alcune pratiche Shinto ritenute superstiziose. In una pesante dichiarazione Hiro Hito (imperatore dell'epoca) nega la superiorità del popolo giapponese, e il suo destino di governare le altre razze, nonché la natura divina dell'imperatore. Ciò quantomeno evidenzia l'esistenza di una radicata idea di superiorità del popolo Giapponese. 
\chapter{Buddhismo in Giappone}

\section{La diffusione del buddhismo}
La religione classica giapponese dei kami è sostituita dal buddhismo a partire dal VI secolo d.C. mediante influenze dalla rielaborazione del buddhismo operata in Cina a partire dal I secolo, che si erano diffuse in Corea e dunque in Giappone. L'introduzione del buddhismo ebbe anche un movente politico in quanto era un modo per avvicinarsi alla Cina, di cui il buddhismo era considerato la più autentica espressione culturale. La struttura sociale giapponese era suddivisa in potenti famiglie, alcune abbracciarono il buddhismo mentre altre la rifiutarono non a partire da critiche concettuali ma perché vedevano i loro privilegi, garantiti dalla religione dei kami, messi a repentaglio dal nuovo ordine sociale buddhista. L'assimilazione del buddhismo avviene dall'alto, la popolazione non è considerata ed infatti inizialmente non si hanno monasteri. A partire dall'epoca Nara il buddhismo è visto come politicamente utile e la città di Nara è una copia della capitale dell'impero Tang.
La diversità del buddhismo era inizialmente destabilizzante, e si cercò inizialmente di piegare il buddhismo alle categorie preesistenti, esagerando gli aspetti familiari. I Buddha e i bodhisattva erano praticamente venerati come kami stranieri e curiosi. Il buddhismo però non poteva essere ridotto a questi schemi e dal VII secolo si hanno testimonianze di eremiti che si interessarono più profondamente al buddhismo, reinterpretandolo liberamente. I templi buddhisti si diffusero gradualmente e a fine VII secolo venne istituzionalizzata la loro gestione da parte dello stato; il buddhismo divenne mezzo del potere imperiale e venne di fatto adottata come religione ufficiale (e dunque politicizzata) al fine di legittimare il potere assoluto dell'imperatore. La spinta ad accorpare la religione dei kami al buddhismo portò a vedere i kami come divinità inferiori ai buddha perché ancora legate al ciclo delle rinascite. In un secondo momento i kami divennero divinità tutelari delle comunità buddhiste, si verificò quanto era avvenuto in india nell'accorpare le divinità induiste nel buddhismo sotto forma dei deva come protettori del Dharma.\\
Nel Giappone medievale analfabeta la diffusione del buddhismo avvenne prevalentemente in forma visiva, tipica era la statua del Buddha, ieratico ed enigmatico. Con il tempo però aumentò la profondità con cui la dottrina era conosciuta dai monaci che cominciarono a rifiutare un'iconografia così diretta, che mira alla devozione piuttosto che alla sapienza. Nonostante ciò il popolo continuò ad essere irrimediabilmente attratto dall'iconografia, che rivelò una grande potenza espressiva. L'immagine del Buddha come uomo-divino era vista dal giapponese nei termini di una convinzione ben radicata nella sua cultura secondo cui il divino è presente nell'uomo; inoltre l'assoluto era legato ad uno stato di tranquillità ed armonia simboleggiato dal sorriso accennato delle statue del Buddha. In questo periodo nasce il monumento della pagoda, una struttura che contiene le reliquie di un illuminato; l'architettura simboleggia la buddhità universale e trascendente.\\
Durante la fondazione del buddhismo in Giappone si aveva la necessità di una figura fondante che segnasse la svolta buddhista della nazione (come Costantino per l'occidente cristiano), venne individuato il  principe Shotoku, reggente al trono a fine VI secolo d.C., come figura mitica devota per la prima volta al buddhismo. La sua figura venne ingigantita e finì per unire le tre vie del buddhismo, confucianesimo e taoismo e l'antico shintoismo. Si narra che fu lui stesso a comporre un celebre commentario di sutra fra cui quello del loto, accomunati dall'ideale della possibilità di salvezza universale tipico del buddhismo mahayana.\\
Dalla fine dell' VIII secolo lo studio del buddhismo si fece più profondo e si pose il problema ermeneutico di quale versione di buddhismo preferire, considerata l'enorme varietà di testi, soggetti tra l'altro ad una rielaborazione cinese che spesso complicava la situazione per l'iniziale inadeguatezza delle traduzioni. Le scritture mahayana, a differenza di quelle antiche, non erano rigidamente organizzate e qualunque testo giudicato ispirato poteva entrare a far parte del canone. Di fronte a queste difficoltà in epoca Nara si istituirono gruppi di studio per approfondire i sutra fondamentali, nacquero così le 6 scuole di Nara che inizialmente interagivano armoniosamente fra loro e che però con il tempo si irrigidirono in sette con identità ben definite. L'esperienza di Nara in cui potere politico e religioso erano simmetrici (anche architettonicamente), fallisce miseramente e nella violenza, tanto che la capitale viene trasferita a Kyoto (in cui inizialmente non si permette la costruzione di templi buddhisti all'interno della città). Nonostante l'esistenza di tali scuole il buddhismo non stabilì mai un canone definitivo nè un'autorità che potesse fissare i dogmi e l'ortodossia. 
\section{Il monachesimo}
La figura del monaco nel buddhismo è centrale, il terzo gioiello dopo il Buddha e la Legge. In lui convive la rinuncia ascetica per dedicarsi a raggiungere l'illuminazione, con la missione sociale della diffusione del Dharma; per questo motivo il monaco vive in una comunità monastica situata vicino le città ed è una figura attiva nella società, nonostante si differenzi nettamente (anche esteticamente) dal cittadino comune. \'E possibile distinguere varie fasi nel monachesimo buddhista in Giappone che non sono da pensare come degenerazioni del primo modello perfetto ma come un continuo tentativo di inverare gli schemi ideologici nel miglior modo per raggiungere l'illuminazione. Si diventava monaci da giovanissimi, si indossava la veste e ci si radevano i capelli per pi fare un rito in cui si rinuncia ai vincoli del mondo. Dopo anni di vita in comunità il monaco riceve la piena ordinazione. La vita del monaco è regolata da rigide regole ed è fatta di silenzio, meditazione, lettura dei sutra  e venerazione di buddha e dei kami. Le regole si fondano sul racconto delle tentazioni di Mara (incarnazione del male) a Buddha, che lo domina e vince. I tipici piaceri terreni sono solamente forme con cui si esprime l'attaccamento alla realtà, che è illusione che porta alla sofferenza. Il sostentamento avveniva inizialmente mediante la questua ed il possesso era condannato; questo modello diventò problematico quando la vita monacale passò dall'essere eremitica e vagabonda al cenobitismo, momento in cui si accettò l'esistenza di un patrimonio collettivo del monastero, sempre basato su donazioni. La donazione era percepita come un atto meritorio ma anche di scambio perché il monaco aveva il compito di svolgere le attività religiose al posto dei laici. In questo modo il Giappone i monasteri si arricchirono, ottennero molti privilegi e diventarono centri culturali. Inizialmente la castità era una regola tenuta in alta considerazione ma con l'affermarsi della corrente Mahayana, che apriva la strada anche ai laici, le norme divennero meno intransigenti. \\
La figura di Mara non è una concessione alla devozione popolare ma incarna un'aspetto profondo dell'esperienza spirituale buddhista: la tentazione del desiderio e delle illusioni che legano al samsara. Mara rappresenta il volto nascosto dell'uomo, ed è la fonte del "peccato" che dal punto di vista buddhista è uno stato mentale transitorio che si può superare con la meditazione e la purificazione personale. Gli insegnamenti del Buddha non hanno come scopo il far diventare buono l'allievo ma di fargli raggiungere l'illuminazione, il male non è dunque negativo perché sofferenza o ingiustizia ma perché allontana dall'illuminazione. La pratica delle virtù morali è uno stadio iniziale del percorso di salvezza ma non è decisiva. Il bene e male nel buddhismo fanno parte dell'illusorietà della realtà, non vi è supremo bene o male assoluto, a differenza che per le religioni monoteiste in cui la priorità assoluta del bene sul male è basilare (visione teleologica ed onnicomprensiva della realtà scaturente da un Dio unico che organizza il tutto nel migliore dei modi per l'uomo, organizzazione razionale). Il fine del buddhismo non è dunque di vincere il male col bene ma di superare entrambi, se moralmente il bene è superiore al male, ontologicamente hanno lo stesso statuto di stati illusori. La soluzione che salva non è la fede in Dio ma la conoscenza della propria natura, che è una realtà di vuoto, al di là del bene e del male. Lo stadio successivo al praticare le virtù morali, che sono semplicemente un mezzo per resistere alle pulsioni più becere, è quello di superare la realtà con la meditazione, raggiungendo un rarefatto stato mentale in cui scompare la percezione di sè e la distinzione fra soggetto e oggetto viene meno. Solo a seguito di ciò il monaco può intraprendere l'ultima fase del cammino, cioè l'apertura mentale verso una realtà ultima, profonda, da vivere a pieno dentro di sé. Il raggiungimento di questa conoscenza superiore è immediato, paradossale. 
\section{Potere politico e comunità monastiche}
La relazione fra potere politico e monastico è sempre stata difficile in Giappone, anche nel primo periodo, in particolare per la contrapposizione con gli ideali confuciani, che rifiutavano l'allontanamento dal mondo tipicamente buddhista e vedevano i riti come superstizioni. Più profondamente, l'esistenza di comunità che non si basavano sui principi su cui era fondato l'impero era destabilizzante poiché fornivano un modo di vivere alternativo che poteva mettere in discussione l'autorità imperiale. Il superamento del bene e del male era inconcepibile per i confuciani, che vedevano in ciò un pericolo per la fermezza con cui è necessario imporre i precetti morali. Nonostante talvolta basate su un'interpretazione ingenua del buddhismo, queste critiche non erano infondate e nella storia si verificheranno casi in cui il buddhismo sfocia nell'apatia morale. Era chiara all'imperatore la pericolosa indipendenza costitutiva delle comunità monastiche ed il fatto che il buddhismo non permetteva in nessun modo di giustificare un controllo unico sul territorio, e dunque farne mezzo di legittimazione del potere imperiale. Emblematico il fatto che il Buddha non designò un successore perché dava importanza solamente all'insegnamento. Per risolvere il conflitto si stabilirono principi di interdipendenza fra la Legge del Sovrano e la Legge del Buddha: il sovrano aveva il dovere di mantenere i monaci che in cambio dovevano mantenere una vita virtuosa ed insegnarla al popolo. A fine VII secolo delle leggi stabilirono i metodi di istruzione dei monaci, relegarono le pratiche rituali all'interno dei monasteri e stabilirono che oltre alle regole del monastero i monaci dovessero sottostare alle leggi dello stato. I monaci erano considerati membri dell'apparato governativo. Un momento importante fu la costruzione da parte dello stato di grandi templi buddhisti su cui era facile esercitare il controllo e che imponevano la condotta e la dottrina a tutti i monasteri, il maggiore fra questi è il tempio Todaiji. Dopo il tentativo di un monaco di usurpare il trono i successivi imperatori strinsero le comunità buddhiste sotto controllo. In Giappone erano penetrati principalmente testi filosofici che non specificavano le norme di comportamento e la pratica di vita del monaco dunque si costituì la scuola Ritsu, affiliata al Todaiji,  per approfondire questi problemi e il governo impose che i monaci fossero istruiti da un maestro di questa scuola. L'organizzazione dei monaci era all'epoca un problema centrale per permettere allo stato di controllarli. La necessità per il buddhismo di legittimarsi istituzionalmente, creando un legame forte con il potere, si ottenne quando il principe Shotoku venne innalzato a figura mitica del buddhismo. Un'alternativa a questo sarebbe potuta essere il maestro cinese Jianzhen, reclutato mediante una spedizione durata 20 anni di due monaci Giapponesi,  che però arrivato in Giappone fu tenuto in poco conto ed aprì una sua scuola che perse rapidamente di vigore. Questa figura sarà in seguito riscoperta e revitalizzata per imporre una ordinazione monacale mista fra la teoria Mahayana e la pratica Hinayana. Questa spaccatura fra il perseguimento della sapienza e l'obbedienza alla legge non prevista al tempo sarebbe diventata in futuro un grave problema: mentre la teoria era basata su un percorso interiore, aperti all'innovazione, la pratica era basata su regole formali e conservatrici. Questa spaccatura rispecchiava due diversi modi di intendere il Dharma: quello tradizionale indiano più centrato sull'importanza del monachesimo, e dunque della legge, e quello dell'Asia orientale, più universalistico, e dunque interessato a trovare una condotta di vita applicabile per tutti, flessibile, che porti alla sapienza. La scelta di un'ordinamento misto da parte di monaci giapponesi prevalentemente Mahayana è dovuta alla necessità pratica di ordinare la comunità sotto regole meno vaghe di quelle Mahayana, inoltre la regolamentazione degli ordinamenti era importante per mantenere un certo livello e credibilità del buddhismo ed evitare l'esistenza di monaci che non conoscessero il Dharma. Con il passare del tempo il sistema misto si abbandonò a favore della linea Mahayana fino a raggiungere azzardi (XIII secolo) secondo cui la via della salvezza avrebbe dovuto svincolarsi dai precetti morali. 
\section{Lo Zen}
In epoca Kamakura si afferma un nuovo tipo di buddhismo destinato ad influenzare profondamente l'intera cultura giapponese. \'E un tipo di pensiero sofisticato ma umile, che si fa beffe di chi crede che grandi teorie possano render conto della verità ultima. La verità ultima non deve passare per i meandri del ragionamento ma è qui e ora, nella realtà così com'è. Educa a vedere l'assoluto nel chiaroscuro del quotidiano. Parlare dello Zen risulta difficile sia per la sua natura sfuggente sia per i discorsi ideologici sorti attorno ad esso nel tempo. Inizialmente, a partire da fine Ottocento, lo Zen è stato idealizzato come spiritualità laica tipicamente giapponese che avrebbe potuto portare verso la Via dei Lumi; decenni dopo, con l'avvento del nazionalismo è stato invece reinterpretato come esperienza mistica che va oltre la ragione (tipicamente occidentale), incarnata dalla figura tradizionale (e un po'romantica) del samurai. Il maestro Suzuki Daisetz nel 1953 sostenne che guardare allo zen da un punto di vista storico è fuorviante in quanto lo Zen ha a che fare con una verità immediata ed eterna, non suscettibile ad un'analisi razionale. Negli anni '60 lo Zen ha incarnato l'ideale di una dolce follia trasgressiva e liberatoria, sono narrazioni ma sono significative perché hanno toccato la sensibilità delle masse, bisogna interessarsene con distacco coscienti della retorica che implicano.
\subsection{Il Chan}
Lo Zen nasce in Cina tra IV e V secolo dalla spinta dei taoisti ispirati dal buddhismo, che rivitalizzava il contesto culturale insterilito dal continuo riferimento ai Classici cinesi. Vi era inizialmente la credenza che il Buddha fosse la reincarnazione di Laozi che tornava dopo un lungo viaggio oltre i confini della terra. Verso la fine dell'VIII secolo l'amalgama di ibridazioni fra taoismo e buddhismo si solidifica nella scuola Chan (in giapponese Zen), conscia della sua peculiare identità. Nonostante si professi un insegnamento "indipendente dalla parola" il testo fondamentale dello Zen è il Ryogakyo, un sutra criptico che punta a sottolineare l'illusorietà della realtà e della mente stessa, "sogno che genera sogni". \'E un testo paradossale che implicitamente si nega, nel tentativo di superare la ragione; fa luce sull'importanza di altri linguaggi, come quello corporeo per raggiungere la consapevolezza della non-dualità fondamentale. Nonostante tutto sia essenzialmente vacuo, visto che il soggetto è in realtà essenzialmente parte del tutto, tutti gli esseri sono illuminati dalla verità ultima.\\
Un problema teoretico è quello della natura della mente umana: è questa intrinsecamente pura, e dunque illuminata dalla verità, o no? In caso affermativo la riconciliazione con taoismo e confucianesimo è semplice poiché andava incontro alla teoria di Mencio sull'innata bontà e quella taoista della purezza della mente come manifestazione del Dao. Questa fu la strada intrapresa, che produsse una rivalutazione della realtà, in quanto fonte di illuminazione. Non solo gli esseri senzienti ma ogni cosa ha una "natura di Buddha", un punto centrale dello Zen è l'identità fra universo e buddhità. Nel tempo si presenta all'interno del Chan la necessità di una istituzionalizzazione, che si ottenne mediante la redazione della "Raccolta della trasmissione della lampada" che forma una tradizione leggendaria ma plausibile che diventerà la storia ufficiale del Chan che risale al Buddha stesso. Nacque così la tradizione del grande BodhiDharma, un monaco venuto in Cina dall'India che meditò per 9 anni nel monastero Shao Lin, che nel tempo si fuse con la figura del Buddha Sakyamuni. Il Chan aveva una narrazione molto basata sulla sua autenticità ed il fatto che derivasse direttamente dalla parola originale, Indiana, senza altre mediazioni asiatiche era di fondamentale importanza. Si diffuse anche una linea di "trasmissione della luce" da patriarca a patriarca per sottolineare questo fatto. Alla fine del VII secolo si verificò una frattura fra Chan del Sud e del Nord a causa della disputa sulla gradualità o immediatezza dell'illuminazione. La prima visione implicava l'importanza della preparazione, e dunque dell'apprendimento mentre la seconda era più interessata all'irrazionale. 
\subsection{Il Rinzai di Eisai}
Le pratiche di meditazione Zen erano presenti in Giappone prima che lo fossero in Cina e certamente prima dell'epoca Nara. Lo Zen si introdusse, inizialmente come pratica meditativa, nella tradizione del Tendai, una scuola giapponese del Buddhismo Mahāyāna fondata da Saichō, discende della scuola buddhista cinese Tiantai, anche conosciuta come scuola del Sutra del Loto. Ma anche grazie ad altri autorevoli maestri cinesi che si trasferivano in Giappone, che però ebbero poca fortuna nel veicolare il profondo messaggio dello Zen, anche perché ancora il buddhismo Mahayana non era abbastanza radicato. Secoli dopo (XII secolo circa), quando in Cina il Chan era al suo apogeo, riuscì ad attecchire anche in Giappone; gli storici fanno iniziare la storia del buddhismo Zen in Giappone con il ritorno del monaco Eisai (1141-1215) dalla Cina, che fece i primi passi che porteranno alla fondazione della scuola Rinzai. Eisai già in età matura decise di intraprendere un viaggio in India con la convinzione che solo ripercorrendo gli autentici luoghi del Buddha avrebbe potuto trovare l'illuminazione, in un periodo di forte inspirazione spirituale. Arrivato in Cina però gli fu impedito di proseguire e si stabilì in un tempio da cui tornò cambiato, tornò dunque in Giappone per insegnare lo Zen. Fu inizialmente avversato dall'establishment (Saicho patriarca del Tendai) ma Eidai si difese sostenendo che lo stesso Saicho professava lo Zen. In realtà la diatriba si centrava sul fatto che Eisai pretendeva di porre lo Zen come unica e vera dottrina, escludendo le altre. Per questi conflitti decise di trasferirsi a Kamakura, dove fu ben accetto per motivi politici, eminentemente per contrastare il potere di Nara da parte della nuova aristocrazia guerriera. Alcuni giudicarono lo Zen di Eisai impuro poiché intriso di influenze esoteriche imposte dal potere politico. Lo zen di Eisai è detto Rinzai e sostiene l'immediatezza dell'illuminazione, era compito del maestro scuotere bruscamente il discepolo per liberarlo dal pensare comune. 
\section{I koan} 
Il koan si presenta come un dialogo in cui un discepolo pone al maestro una domanda sulla dottrina e il maestro gli risponde in modo enigmatico, paradossale, ironico ma senza nessuna spiegazione. Nasce come strumento di meditazione del Chan Cinese in epoca Tang ma ha radici ben più antiche. Le fondamenta dei koan si possono rintracciare nel mondo religioso induista, nei veda e nell'Upanishad, nella tradizione delle lotte rituali di enigmi tra brahmani (membri della casta sacerdotale induista). In un clima di forte tensione antagonistica lo sfidante poneva l'enigma, l'altro doveva scioglierlo con la sua risposta per poi contrattaccare con un altro enigma fin quadno uno dei due non sapesse più rispondere ed era sconfitto. Gli enigmi potevano essere linguistici, logici o di sapienza, legati alle dottrine metafisiche. Solitamente la domanda era molto complessa mentre la risposta semplice e nonostante il gioco fosse essenzialmente banale era caricato di una forte tensione e angoscia perché in gioco c'era l'autorità e talvolta la vita dello sfidante. Questa prassi si trasferisce in Cina e si intreccia con l'enigma taoista. Nei monasteri buddhisti si diffonde la pratica del "mondo" in cui gli allievi possono porre brevi domande dottrinali al maestro; ma comunicare con il linguaggio la verità è problematico se si crede che il linguaggio è relativo e non può esprimere la realtà ultima. Il linguaggio può però avvicinare la mente al "vuoto" se portato ai suoi limiti, ribaltando la pratica del "mondo" nascono dunque i koan. Se nel mondo il maestro si doveva impegnare ad articolare un discorso chiaro (pur consapevole della limitatezza del linguaggio), nel koan è l'allievo a doversi sforzare a decifrare le frasi del maestro mediante lunghe meditazioni. La meditazione però non è mediata dalla razionalità, che deve essere superata per attingere alla verità più profonda, per andare oltre le illusioni. Talvolta il koan è ricorsivo, presenta cioè un circolo vizioso su cui l'allievo deve meditare. Il koan non è da intendersi come un indovinello che stimola la fantasia ma come un gioco rischioso che mette alla prova il pensiero comune, distrugge la logica e apre varchi verso l'esperienza della realtà profonda.  
\end{document}