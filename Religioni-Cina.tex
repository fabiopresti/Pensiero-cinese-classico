\documentclass[10pt,a4paper]{report}
\usepackage[top=1in, bottom=1in, left=1.2in, right=1.2in]{geometry}
\usepackage[utf8]{inputenc}
\usepackage[T1]{fontenc}
\usepackage[italian]{babel}
\usepackage{amsmath}
\usepackage{amsfonts}
\usepackage{amssymb}
\usepackage{makeidx}
\usepackage{placeins}
\usepackage{graphicx}
\author{Fabio Prestipino}
\title{Religioni e filosofie della Cina}
\begin{document}
	\maketitle
\chapter{Introduzione: la Cina}
\section{Cronologia essenziale}
\begin{itemize}
	\item Dinastia Xia 2100 a.C.-1600 a.C. (quasi leggendaria, poche testimonianze)
	\item Dinastia Shang 1600 a.C.-1046 a.C.
	\item Dinastia Zhou occidentale 1045 a.C.-771 a.C.
	\item Dinastia Zhou orientale 771 a.C.-256 a.C. diviso in: "Periodo delle primavere e degli autunni" 722 a.C. al 481 a.C. e "Periodo dei regni combattenti" 453 a.C.-221 a.C. 
	\item Dinastia Qin 221-206 a.C. (nasce l'impero)
	\item Dinastia Han 206 a.C.-220 d.C. (buddhismo arcaico)
	\item Periodo dei tre regni 220 d.C.- 280 d.C.
	\item Dinastia Jìn 280 d.C.- 480 d.C. (dopo l'invasione migrano a sud)
	\item Invasione della Cina settentrionale dei turco-mongoli 311 d.C. Divisione in impero del nord e del sud (proliferazione del buddhismo)
	\item Riunificazione sotto la Dinastia Sui 589 d.C. (Epoca d'oro del buddhismo)
	\item Dinastia Tang (618 d.C.-907 d.C.)
\end{itemize}
\section{Tradizione}
A differenza della cultura greca che cerca di trovare giustificazione delle proprie fondamenta la cultura cinese parte da un sostrato accettato che non si mette in dubbio: la tradizione; gli aspetti teoretici non sono fondamentali (Confucio: "io trasmetto senza creare nulla di nuovo"). \'E un sapere che procede cumulativamente, lasciando domande in sospeso per generazioni, risulta infatti difficile parlare di filosofia, che è legata al logos greco (la parola cinese per "filosofia" è un neologismo mutuato dal giapponese). Gli scritti non sono autoconsistenti ma si inseriscono in una rete di relazioni e rinvii alla tradizione. Manca la dialettica, la contraddizione non è problematica ma è vista come alternativa possibile (yin e yang). L'obiettivo non è la gratificazione intellettuale ma la santità, la capacità di condurre una vita migliore, in armonia con il mondo.
\section{Lingua e segni}
La mancata ricerca del teoretico può essere vista in relazione alla scrittura, ritenuta di origine divinatoria, con poteri magici, è il segno stesso a far partire il pensiero e non al contrario il linguaggio che si fonda su costruzioni concettuali. Il segno stesso è percepito come una cosa fra le cose, e fa scaturire un pensiero che si inscrive nella realtà e non al di sopra di essa. Si consideri inoltre l'importanza della struttura grammaticale del greco e del latino nello sviluppo della filosofia occidentale (gli scolastici prendono le mosse dalle categorie della grammatica latina), in cinese il significato è solamente costruito dalla posizione del segno nella frase e non si ha una struttura di base soggetto-predicato (si pensi alle conseguenze sul concetto di vero o falso, strettamente legato allo schema soggetto-oggetto sin da Aristotele). Manca il verbo essere, peculiare del Greco. L'epistemologia e la logica sono dunque impossibili nella Cina arcaica! Non è neanche possibile pensare un parallelismo tra struttura razionale della mente umana e realtà, che sta alla base del pensiero occidentale.
\section{Conoscenza e azione: il Dao}
Ne risulta una riflessione più legata all'azione che alla conoscenza in sè. Questo fatto si configura in due modi: 
\begin{itemize}
	\item Confucianesimo: azione come orizzonte della conoscenza, conoscenza garantita dall'azione. 
	\item Taoismo: negazione di ogni rapporto fra conoscenza e azione. 
\end{itemize}
L'azione è misura del discorso, almeno fino all'avvento del buddhismo.\\
La verità ha connotazione etica (perché vista come discorso che si riflette in un'azione corretta, saggia). Dao vuol dire "strada", "cammino", ma anche "parlare" ed è un termine ampiamente usato per indicare l'insegnamento pratico che un pensiero vuole veicolare, la strada di vita da seguire proposta da un certo pensiero.
\section{Soffio o Qi}
Il pensiero cinese ha fiducia nella possibilità dell'uomo di abbracciare la totalità del reale, che è unica, oltre la molteplicità delle parti. L'uomo non si concepisce fuori dal mondo, si crede che esista un Qi (soffio), un'energia vitale che anima l'universo, un dualismo mente-corpo è impossibile poiché ogni aspetto della realtà, fisica e mentale, è Qi (è come se fosse il tramite fra realtà unica ed indeterminata e molteplicità). Dal Qi discende la morale, l'estetica e tutti i valori, ma non è una nozione astratta ma anzi è percepita quotidianamente e carnalmente. 
Al posto della causalità trova un ruolo centrale il modello generativo, per cui l'opposizione fondamentale non è trascendete/immanente ma virtuale/manifesto. Si rifiutano i dualismi Dio/mondo, Bene/Male, essere/nulla e si preferisce la coesistenza, la visione del mondo non è come un insieme di entità indipendenti ma come una rete di relazioni. Inoltre alla generazione ex nihilo si preferisce la ciclicità, in questo modo il problema dei fondamenti, incarnantesi ad esempio nel problema del Dio creatore occidentale, non figura con la stessa preponderanza nel pensiero cinese. Il soffio è visto sia da taoisti che confuciani come il mutamento, i confuciani lo vedono come "vita che genera incessantemente la vita" mentre i secondi come Vuoto, inteso come virtualità per eccellenza, che ha il vantaggio di essere indeterminato e che quando si definisce nella molteplicità si irrigidisce e deperisce.\\
Quando il pensiero cinese assume una connotazione più astratta e metafisica (soprattutto con il taoismo) il Dao è concepito come il flusso naturale, inconoscibile, che va oltre l'esperienza, della realtà nel suo complesso, mentre il Qi è ciò che lega il Dao a ciò che è manifesto.  
\section{Relazione e Mezzo}
La relazione non è un mero legame fra entità precedentemente distinte ma costituisce gli esseri nella loro esistenza. Confucio cala questa idea nella società, affermando l'importanza delle relazioni che caratterizzano l'umanità. L'uomo è visto come "ciò che nasce fra" gli opposti, e che li lega, dissolvendo l'opposizione, negata dal pensiero cinese; l'uomo partecipa attivamente alla realtà per portare a compimento l'opera cosmica, in questo consiste l'eticità. Il concetto di Mezzo (zhong) è centrale ed è difficile da rendere: come sostantivo indica la via giusta, come verbo è la freccia che colpisce il bersagli nel centro. Lontano dal mantenere un "giusto mezzo" fra due estremi o di un freddo compromesso, il zhong è la tensione verso il miglioramento, verso l'alto. Il Mezzo è in stretta relazione al Dao, perché il Dao insegna la retta via per agire in vista del miglioramento, spinta dalla tensione del Mezzo.   
\section{Cultura arcaica da Shang a Zhou}
L'antichità cinese corrisponde con le dinastie Xia (quasi mitologica, III millennio a.C.), Shang (storica, dal XVIII secolo a.C.) e Zhou (XI secolo a.C.) che però sono da pensare non come succedentesi cronologicamente ma come sviluppantesi contemporaneamente, a partire da un tronco comune, e prevalendo l'una sull'altra in certi periodi storici. Gli Zhou provengono dai margini della Cina, sono più rozzi e governano fino al III secolo a.C. quando i Qin fondano l'impero. Il periodo Zhou è diviso in orientale e occidentale, il primo è visto come un'età dell'oro dal punto di vista politico da parte dei confuciani. Un fatto interessante è che il potere era basato su un sistema religioso incentrato sul re Zhou e la sua famiglia, i feudatari erano membri della famiglia reale e avevano il diritto di fare un culto per il fondatore della propria casata, legando il potere politico ai culti familiari. Si crede che la concezione di stato come famiglia, tutt'oggi forte in Cina, affondi le sue radici da ciò. La gerarchia era Cielo-Re-vassalli (in ordine di prossimità familiare al re). Di questo periodo è la concezione divinatoria dei segni, da cui scaturisce la razionalità cinese. Infine, nel passaggio dagli Shang ai Zhou si passa dalla nozione personale di Primo-Antenato come divinità ad una cosmologizzazione del Cielo che regola i processi cosmici, fra cui quelli umani. Il sovrano non è più un discendente della divinità suprema (antenato divinizzato) ma è scelto per mandato del Cielo.\\
Del periodo arcaico sono le prime testimonianze scritte di simboli divinatori su ossa e gusci di tartaruga trasmessi dalla divinità e da interpretare; nasce un tipo di razionalismo divinatorio, secondo cui vi è un'ordine superiore nel responso oracolare, che deve essere seguito. A differenza dell'oracolo Greco quello cinese è chiaro, e talvolta risponde a domande semplici anche dicotomicamente con Si/No, non si ha bisogno di trance o estasi, il dialogo umano-divino parte per iniziativa razionale dell'uomo e gli dei rispondono.\\
La religione Shang era basata sul culto degli antenati e delle potenze della natura a cui si offrivano sacrifici. Gli antenati sono spiriti che dimorano nel mondo dei morti e che permettono la mediazione con le forze sovrannaturali perché mantengono un legame con la discendenza vivente. Ci sono pochi miti perché (oltre per il deliberato occultamento confuciano) gli antenati con il passare del tempo perdono la loro individualità, diventando semplicemente un rango da rispettare. La natura in parte familiare del culto ha paradossalmente sfavorito l'antropomorfizzazione delle divinità, che sono percepite come garanti dell'ordine familiare su cui si fonda l'armonia, a differenza del culto Greco. \\
Il culto dei morti era molto preciso ed elaborato, quasi burocratico, la divinazione inizialmente aveva la funzione di assicurarsi che il sacrificio sarebbe stato accettato e non ad interrogare gli spiriti è come una domanda retorica la cui risposta si ottiene mediante segni che l'indovino sa cogliere (spesso sotto forma di segni su ossa). Un tratto peculiare è la comunicazione diretta e sempre possibile fra ambito umano e sovrannaturale. L'analisi degli oracoli evidenzia come la Cina arcaica credesse nell'esistenza di una divinità superiore agli spiriti naturali e agli antenati che popolavano la realtà, questa figura si sviluppa nel periodo Shang quando gli ultimi sovrani, con pretese di apoteosi, si identificano come discendenti di "di", questa divinità superiore (fatto che sarebbe entrato in profondità nella cultura cinese). Nel periodo Shang solo l'imperatore può svolgere il culto degli antenati suoi e di tutta la comunità, in qualità di sacerdote per tutti, ne segue che non esiste una casta sacerdotale e questo porta ad un'identificazione di religione e politica. La divinità suprema non è un creatore onnipotente primo motore ma un principio di ordinamento armonico del mondo, senza una connotazione razionale (dell'armonia fanno parte gli spiriti ed il sovrannaturale). Nel passaggio da Shang a Zhou la concezione religiosa vira verso la cosmogonia, e la divinità suprema diventa il \textbf{Cielo}. Ciò comporta un passaggio da una concezione del sovrannaturale fuori dal mondo ad un sovrannaturale che sta nel mondo. Il rapporto fra il re e la divinità non è più di parentela ma di "mandato" che il Cielo, garante dell'ordine e dell'armonia, affida al sovrano. Il passaggio dagli Shang agli Zhou è giustificato dalla volontà del Cielo di ristabilire l'armonia che gli Shang non sapevano più assicurare. Nel XIX secolo la parola "rivoluzione" è tradotta come "cambio di mandato". Da ciò notiamo quanto spiritualità e riflessione astratta siano legati alla vita umana e alla politica: l'ordine del Cielo e quello politico sono legati. Il passaggio da Shang a Zhou segna anche un mutamento della struttura sociale da puramente piramidale a feudale, dove le potenti famiglie diventano sempre più rilevanti.\\
Se il greco è un vasaio che modella a suo piacimento l'argilla, il cinese è un lapidaio che lavora la giada seguendo le naturali forme che questa ha già in sè.
Parliamo di due omofoni del termine LI: rito sacrificale oppure ordine rituale. La prima accezione di questa parola è stata modificata (slittamento semantico) da Confucio che nei Dialoghi la usa con senso di "spirito rituale", che come vedremo perde la connotazione religiosa e sacrificale. La seconda accezione invece in origine indicava le venature della giada, significa anche "ordine" inteso nell'accezione rituale e non razionale, che rispecchia perfettamente l'esempio del vasaio e del lapidaio. La razionalità cinese non emerge in contrapposizione al mito come in Grecia ma in seno allo spirito rituale. In questo modo la cultura cinese si fa portatrice di senso, poiché l'uomo è parte dell'ordine supremo e il suo compimento sta nel contribuire a realizzarlo.\\
Schematizziamo il passaggio Shang-Zhou come segue:
\begin{itemize}
	\item Concezione del potere: non più imperatore discendente della divinità ma scelto per mandato celeste (cosmologizzazione). SOvrano non ancora imperatore
	\item Società: passaggio da società piramidale basata sulla famiglia imperiale a feudale.
	\item Sacrifici: prima svolti solo dal capofamiglia del clan dominante, diventano il culto degli antenati che tutti i nuclei familiari praticano
	\item Pratica divinatoria: prima praticata solo a corte e ad uso politico, poi estesa a tutti ed utile per sapere cosa fare in ogni circostanza. La divinazione, in termini cosmologici, serve a svelare la trama dell'ordine nascosto del cosmo. 
\end{itemize}
L'epoca degli Zhou occidentali è idealizzata come età dell'oro dai posteri; nel 770 a.C. gli Zhou indeboliti si devono spostare a oriente (epoca degli Zhou orientali), comincia un periodo di tumulti detto delle Primavere e degli Autunni e degli Stati Combattenti. I feudatari si proclamano indipendenti ed infrangono le antiche norme proclamandosi re. Con il declino dell'egemonia del sovrano si affinano i discorsi filosofici che risuoneranno in tutta la storia del pensiero cinese. 
\chapter{Confucianesimo e Taoismo in Cina}
Si ponga attenzione alla differenza fra Confucio e confucianesimo, quest'ultimo è frutto di un dialogo ed una reinterpretazione del nocciolo degli insegnamenti confuciani che può presentarsi molto lontano dall'originale. Subito dopo la morte di Confucio la sua figura è divinizzata, in epoche successive gli si attribuisce erroneamente la scrittura dei classici, che al tempo di Confucio invece costituivano un sapere orale, fluido, al più trascritto in frammenti su listarelle di bambù. Confucio, profondo conoscitore di questo sapere, lo utilizzò e modificò ma per una sistematizzazione in forma di rotoli di seta (e successivamente cartacea) bisognerà attendere il monumentale lavoro intrapreso in epoca Han, anche col fine politico di dare un fondamento ideologico al nuovo impero. .
\section{Confucio ( 551 a.C. – 479 a.C.)}
Quando nel periodo delle Primavere e degli Autunni la casa Zhou comincia a disgregarsi (e poi con la fuga ad Oriente del sovrano Zhou) ci si pone il problema di come il Cielo abbia potuto permettere che il trono fosse mantenuto da una casa in decomposizione, questo mette in moto il pensiero filosofico (come il Platone il movente era stata la crisi dell'ordine politico della polis e della concezione del mondo che portava). Confucio è la figura centrale della cultura cinese al pari di Cristo o del Buddha, ed è la figura che segna l'apertura alla filosofia di una delle 4 "civiltà assiali" (Cinese, Indiana, Greca ed Ebraica), che curiosamente avvengono nello stesso periodo. Egli si fa interprete della crisi che investiva la società cinese dell'epoca, propone una via per restaurare un ordine sociale e al contempo propone una società diversa, in cui le qualità morali prevalgono su quelle ereditarie. Non è contrario alle gerarchie ed è un rappresentante della classe sociale emergente degli shi. L'attenzione di Confucio è sull'uomo, si dice che il suo pensiero costituisca una grande \textbf{scommessa sull'uomo}, cioè un'ottimistica speranza nella possibilità di perfezionamento di ognuno. L'interesse per il pensiero metafisico è scarsissimo: "Si possono udire dal Maestro lezioni di civiltà, ma la sua parola intorno alla natura dell'uomo e alla Via del Cielo non è dato udirla". L'idea è che questi sono al di là dell'umana comprensione e interrogarvisi distrae solamente dal vero obiettivo che è perfezionare se stessi e realizzare un'armonia immanente. Sarà accusato d'empietà da Mozi per questo.
\subsection{Dati biografici}
Il nome originalmente vuol dire Maestro Kong, sulla sua figura sappiamo poco, principalmente dai "Dialoghi", scritto sulla base di appunti dei suoi allievi. Nasce a Lu e vive 72 anni (è infatti rappresentato come un vecchio saggio), è vicino alla casa Zhou ed è vicino ai suoi valori, fa parte di un ceto in ascesa, gli \textbf{shi}, intermedio fra nobiltà guerriera e contadini/artigiani, classe che in età imperiale sarà la detentrice della cultura. Proviene dalla bassa nobiltà ma ha un'educazione umanistica di alto livello, studia le 6 arti:  materie astratte ma anche la pratica fisica come il tiro con l'arco ed una coltivazione della persona con la calligrafia. Studia anche dei testi fluidi, che costituiscono un sapere comune, e che diventeranno i classici cinesi, utili per arricchire il proprio vocabolario e quindi la sensibilità sulla realtà (Classico delle odi) e imparare i riti (Classico dei riti), studiare la tradizione (classico dei Documenti), avere una visione comprensiva della realtà (classico dei Mutamenti). \'E impegnato in politica e fa il ministro della giustizia, deluso dal malgoverno del sovrano rinuncia alla carriera politica e inizia una peregrinazione di 12 anni per i vari principati in cerca della Via. Si propone di dar consiglio ad altri sovrani ma senza successo, in vecchiaia fa ritorno a Lu ed insegna ai discepoli. 
Il suo insegnamento è articolato in tre poli:
\begin{enumerate}
	\item L'apprendimento
	\item Lo spirito rituale
	\item Il senso di umanità
\end{enumerate}
\subsection{L'apprendimento}
 Dai Dialoghi non si evince un pensiero sistematico ma sembra si voglia veicolare il modo in cui si diventa integralmente esseri umani. La centralità nell'uomo e la convinzione \textbf{ottimistica} che questo sia continuamente perfettibile mediante l'apprendimento è rivoluzionaria in una società rigidamente suddivisa e aristocratica (perché implica un'uguaglianza di coloro che decidono di apprendere). Il pensiero di Confucio non propone una dottrina e non è intellettualistico ma si configura come una esperienza di vita di cui l'apprendimento è il primo passo. L'apprendimento è fonte di gioia ed è in sè giustificato, a prescindere dai riconoscimenti che porta. La conoscenza non ha funzione di sovvertire l'ordine gerarchico mediante l'ascesa sociale ma serve per ricoprire il proprio posto nel modo migliore e governare gli altri per il loro maggior bene. La nobiltà in Confucio non è più legata alla nascita ma soprattutto dal valore che si ha in quanto esseri umani completi, al valore morale; si apprende per diventare uomini di valore (ci si soffermi sul peso di quest'idea nel contesto sociale del V secolo a.C. in Cina). L'umanità non è un dato ma qualcosa che si costruisce nelle relazioni e nell'armonia di una comunità, umani non lo si è mai abbastanza e bisogna tendere a diventarlo. 
\subsection{Ren: l'umanità}
La grande idea di Confucio sta nella nuova concezione di umanità, in termini di relazione. La parola \textbf{Ren} anticamente era la "paternalistica benevolenza" del sovrano"; in Confucio indica più in generale l'idea di umanità ed è formata dagli ideogrammi per "uomo" e "due", che evidenziano la natura relazionale di questo concetto. Qui si nota la differenza con il pensiero occidentale: non è la moralità, esterna all'uomo, che definisce il modo migliore di instaurare rapporti, ma è il legame morale che costituisce la natura stessa dell'essere umano. Confucio non riconosce il Ren a nessuno tranne che ai mitologici santi dell'antichità; questo non deve essere visto come un'ideale rigido e stereotipato a cui conformarsi, ma un qualcosa di non ben definito a cui tendere all'infinito. Confucio non definisce con esattezza il Ren per non limitarlo (si noti la lontananza da Socrate). L'amore fra esseri umani proprio del Ren fu accostato dai missionari all'amore cristiano ma in realtà ha una connotazione molto più terrena ed emozionale, basata sulla reciprocità dei rapporti umani. Il principio per raggiungere il Ren è la mansuetudine, la capacità di mantenere un \textbf{giusto Mezzo} (zhong), concepito come equilibrio dinamico, come quello di un funambolo. A partire dall'equilibrio nella propria persona si ottiene l'armonia anche con gli altri (si tenga a mente che l'umanità del Ren è essenzialmente relazionale), a partire dalla famiglia e poi via via con la società intera. Una massima fondamentale, analoga a quella cristiana è "Ciò che non vuoi sia fatto a te, non farlo agli altri" (Gesù, Kant e molti altri in ebraismo ed islam). Non si ha però egualitarismo in Confucio, il Ren si inserisce nel rispetto della gerarchia. La relazione umana prototipica è quella padre-figlio e della pietà filiale, questa è il modello su cui si basa la sana gerarchia ed il rapporto principe-suddito. Anche le altre relazioni familiari (fratello-fratello, marito-moglie,...) sono considerate fondamentali dai Confuciani, tutti sono basati sulla fiducia (xin $\simeq$ cuore/anima) che può scaturire solo dall'integrità. Idealizzazione della famiglia.\\
I tre ingredienti per essere un uomo di valore (junzi) sono:
\begin{itemize}
	\item zhong: giusto mezzo, in senso dinamico
	\item xin: affidabilità
	\item shu: Reciprocità 
\end{itemize}
\subsection{Lo spirito rituale}
La relazione che rende umani e che sta alla base del Ren ha carattere rituale, il comportarsi umanamente coincide con il seguire il rituale (per raggiungere il Ren bisogna "vincere il proprio io per rivolgersi ai riti"). Il termine LI, in accezione rituale (di cui sopra), è la parola più frequente nei Dialoghi. Dall'idea dell'eticità della ritualità segue anche un'ideale estetico che incarna l'armonia del rito, la bellezza formale dei gesti, e che è perfettamente incarnato nell'armonia musicale. Il ritualismo del LI non è mera etichetta poiché la bellezza formale coincide con la sincerità degli intenti; nel periodo di Confucio il rito ha già perso il vigore arcaico e lui vuole ristabilirlo. Confucio opera uno slittamento semantico del termine LI dal senso di sacrificio rituale a quello di rito interiorizzato e fortemente sentito e cioè la capacità di comportarsi in maniera appropriata in ogni circostanza. Il significato slitta dal religioso all'umano. Per Confucio il rapporto con la dimensione etica non può esaurirsi ad un atto di sacrificio, la vita intera deve essere un rito nel senso di mantenere sempre una \textbf{postura} solenne e sentita, la moralità che offre Confucio non è data da leggi ma da un atteggiamento alla vita. Gli aristocratici finiscono per essere junzi perché riescono a vivere ritualmente.\\
Ci si soffermi sul rapporto fra rito e umanità: la differenza fra animale e umano è che negli umani i sentimenti sono ritualizzati e acquisiscono un significato smettendo di essere puro istinto. Mediante l'adesione al rito è possibile mettere la propria individualità in secondo piano a favore degli altri. Il riferimento alla trascendenza tipico delle civiltà occidentali è sostituito dal riferimento alla tradizione, che non è mai sterile ripetizione ma ogni volta sentita personalmente. Mediante la ripetizione il rito forma abitudini; anche per Aristotele la virtù morale deriva dall'abitudine. Un fatto interessante è che essendo convenzioni, i riti possono cambiare nel tempo, in questo senso il Confucianesimo non è necessariamente tradizionalista nel senso più basso del termine.
\begin{quote}
	 Il maestro disse: "Se non vi è il rituale a regolarle, la gentilezza si fa molestia, la prudenza si fa viltà, l'audacia si fa ribellione, la rettitudine si fa intolleranza.
\end{quote}
Moralità confuciana come quella kantiana, non c'è un corpus di leggi da seguire ma un atteggiamento generale da seguire, che diventa abitudine virtuosa e forma il destino di un'uomo, che si adatta al contesto ed alla persona, condividono anche la regola aurea.
\subsection{L'uomo di valore e il suo Dao}
Lo scopo umano è per Confucio di elevare la propria umanità e questo scopo è sacro, la saggezza non consiste solo nel rispettare le divinità ma anche nella la moralità individuale. Seguire il proprio scopo, cioè seguire la Via (il Dao) è sacro; Confucio dice di aver conosciuto a 50 anni il "decreto del Cielo" (come il decreto che il Cielo affida all'imperatore Zhou) che lo spinge a seguire il Dao.
Il Dao confuciano si presenta come un'aspirazione ad un ideale di armonia che ogni junzi (lett. figlio di nobile) deve perseguire \textbf{perfezionando se stesso continuamente}. L'uomo di valore, il \textbf{junzi}, è tale non per discendenza ma per virtù: egli segue il Dao rispettando i tre pilastri dell'apprendimento, senso d'umanità e spirito rituale, non solo nella vita individuale ma anche nella famiglia (estensione dell'individuo) e nella politica (estensione della famiglia). Il sovrano è dunque visto come un junzi che fa il bene dello stato seguendo la moralità del Dao, in questo Confucio fa slittare il principe a uomo esemplare e il mandato del Cielo a missione morale; in generale porta il divino verso l'umano. In questa concezione il sovrano possiede la virtus (da intendersi in senso latino come capacità di imporsi in modo naturale, con carisma) che gli permette di stabilire l'armonia e la moralità nel popolo senza violenza. L'apprendimento ha sul piano politico la stessa centralità che ha su quello individuale: il sovrano deve educare il popolo. L'educazione avviene mediante esempio e imitazione di modelli e non tentando di conformarsi a principi a priori. Il junzi è contrapposto all'uomo dappoco, letteralmente "piccolo uomo" nel senso di piccole dimensioni, sottosviluppato perché non ha coltivato la sua umanità.\\
Il sovrano deve essere carismatico e con la sua virtus si deve imporre senza violenza (Confucio cita Shun), il trono guarda sempre a sud perché è la direzione in cui, secondo la mistica cinese, lo Yang (energia maschile) è più forte, cioè la sua virtù è massima. Fatto interessante è che il sovrano non è virtuoso in quanto agisce e gestisce il regno in modo efficiente, ma anzi "agisce senza agire", cioè fa il necessario, fornendo l'esempio. Sovrano come stella polare attorno a cui girano tutte le stelle. La città proibita è detta "purpurea" perché la stella polare è rossastra.  
\subsection{Rettificare i nomi}
In un passaggio dei Dialoghi Confucio afferma che se fosse lui a governare la prima cosa che farebbe sarebbe quella di rettificare i nomi, cioè di adeguare i nomi alla realtà "bisogna trattare il sovrano (chi? il nome che designa il sovrano) da sovrano (nome che indica il ruolo reale)". Per inverare questo adeguamento bisogna agire sui nomi, dando nomi solo a ciò che lo merita, e sulla realtà, cercando di far coincidere le cose ai nomi. Questo ideale rispecchia la tarda speranza di Confucio in un mondo che si autoregola senza la necessità di un governo, capace di armonizzarsi da solo. In questo il modello era il mitico sovrano Shun che si limitava a stare seduto a sud e a non-agire (stampo taoista, come vedremo). L'ideale ultimo di Confucio è una armonia spontanea fra uomo e natura che si inveri senza bisogno di insegnamenti ("Vorrei tacere").\\
Nella civiltà cinese la scrittura ha un ruolo centrale, legati alla figura di Confucio sono i "testi canonici" che ha scritto lui stesso o che sono anteriori ma che ha rimaneggiato per usarle nei suoi insegnamenti. Fra i testi canonici i più importanti sono:
\begin{enumerate}
	\item Il classico dei Mutamenti
	\item Il classico delle Odi
	\item Gli annali delle Primavere e degli Autunni
	\item Il classico della Musica
	\item Il classico dei Riti
	\item Il classico dei Documenti
\end{enumerate}
Questi testi formano un corpus culturale comune paragonabile ai testi sacri occidentali ma molto diversi da questi in quanto mancanti di una rivelazione, di organicità; in particolare per la loro forma non ammettono il concetto di eterodossia ed eresia. Come detto, questi testi partecipano al passaggio da una cultura di tipo divinatorio ad una di tipo cosmologico. Il testo e i simboli saranno sempre concepiti come magici in quanto rispecchiano una trama fondamentale di cui è tessuto l'universo, prima ancora di rappresentare un discorso sull'universo. I Classici sono da intendersi in questi termini, questi non sono legati ad un autore ma sono il punto di partenza da cui tutti gli autori prendono le mosse. Nei secoli successivi l'influenza di Confucio non sarà forte solo per le correnti di pensiero confuciane ma per tutte le scuole (sorte soprattutto nei secoli degli stati combattenti) che si propongono di instaurare un canone, soprattutto per il rivoluzionario uso della letteratura non a fine prettamente rituale ma anche didattico, profano. 

\section{Mozi (V secolo a.C.)}
Vive nel contesto culturale e storico della prima parte del periodo degli Stati Combattenti, continua e al contempo critica l'umanesimo confuciano con un discorso propendente verso la razionalità, caratteristica tipica dei pensatori di questo periodo. Per comprendere i testi cinesi di questo periodo bisogna innanzitutto capire a chi si riferiscono e chi criticano. 
\subsection{Biografia e contesto sociale}
Vive tra la morte di Confucio e la nascita di Mencio in uno stato centrale, come Confucio, ma viene da un contesto molto differente da quello del maestro: si crede sia un artigiano per i vari riferimenti alla sua abilità tecnica e nei suoi scritti, a differenza che nei Dialoghi la sua figura è quasi assente. Ha un'approccio più pratico ed utilitaristico di Confucio. Studia autonomamente la dottrina confuciana e come il maestro viaggia per il paese offrendo il proprio sapere ai principi, ma propone un sapere di tipo pratico, in risposta al crescente bisogno di manodopera tecnica e dell'ascesa del ceto sociale degli shi. Dal punto di vista sociale è fortemente critico dei privilegi per nascita dell'aristocrazia feudale e valorizza la "promozione dei più capaci". Già Confucio aveva privilegiato la qualità morale alla nobiltà di nascita ma in Mozi l'uomo di valore è sostituito dall'uomo capace, il discorso è più incentrato sulla pratica e la tecnica. Il legante della comunità per Mozi non è la famiglia come in Confucio; i moisti non si sentono un'elite morale come i confuciani e sono una comunità fortemente strutturata, con a capo un gran maestro.\\
\subsection{L'utilitarismo}
Mozi rifiuta esplicitamente l'autorità (principe, maestro), e deve dunque gettare delle fondamenta razionali al suo pensiero e, dato che la razionalità deve scevra da caratteri soggettivi, nel Mozi si perde la centralità dell'autore. C'è ossessione di giustificare la fondatezza delle affermazioni, da qui l'interesse verso la logica. Non si deve però pensare che la ricerca di fondamenta e la razionalità siano di tipo occidentale, infatti leggendo il Mozi appare chiaro che le fondamenta vanno ricercate nella tradizione e non nell'obiettività, continua ad essere assente il discorso epistemologico. I criteri di giustificazione sono di ordine pratico e non vogliono essere adeguati alla realtà. I tre principi della giustificazione sono fondamento (nei re dell'antichità), origine (testimonianze del popolo), utilità (pratica e politica). Un esempio lampante dell'applicazione del criterio di utilità è relativa ai riti funebri, secondo i moisti, a differenza dei confuciani, il lutto non deve essere portato per tre anni perché troppo dispendioso economicamente. Anche le spese inutili come quelle per il lusso tipiche dell'aristocrazia feudale sono rifiutate, tratto distintivo di una classe che ha ottenuto la sua posizione sociale con il sudore. Anche la musica è giudicata inutile, è qui palese il distacco dall'ideale di armonia confuciano, i moisti portano l'utilitarismo al parossismo.
\subsection{La politica}
L'ideale sociale è quello di coltivare l'utile della comunità, spinti da un'"amore universale" verso i propri simili, nel moismo l'uguaglianza e l'assimilazione dei simili è fondamentale. In questo Mozi si avvicina all'ideale di ren confuciano ma ne sottolinea l'aspetto dell'uguaglianza degli uomini più che l'armonia dell'io-famiglia-stato confuciano poiché Mozi va contro il nepotismo che può scaturire dall'anteporre l'amore per chi ci è prossimo. Effettivamente la corruzione è un lato oscuro del confucianesimo in politica che ha afflitto al Cina sin dai primordi. Il ren confuciano si radica nel sentimento mentre l'assimilazione moista sull'obiettività. L'appello alla ragione invece del sentimento rispecchia la mancanza di fiducia nella naturale bontà dell'animo umano, che non deve ricercare la sua natura ma deve essere guidato dalla ragione verso il bene (interesse) comune.\\
Vi è dunque una visione pessimistica dell'umanità, contrapposta a quella di Mencio, e si ha una percezione delle origini come di un periodo barbaro in cui l'uomo nel suo stato naturale non era governato e vigevano diverse nozioni di moralità. Con il formarsi del governo si impone un unico "senso del giusto" che, analogamente all'"amore universale", viene dal Cielo. Non si pensi però che Mozi era un democratico: dalla nozione meritocratica della "promozione dei più capaci"  fa seguire l'ideale di "conformità ai superiori", in modo da garantire un ordine gerarchico. Ma, alla fine della piramide, chi assicura che il re segue il principio dell'"amore universale"? Il Cielo stesso lo garantisce: questo è concepito come un'essere senziente che può scatenare la sua ira sul re se non rispetta questi principi universali. Il decreto celeste che vede un sovrano che naturalmente agisce benevolmente perché parte dell'armonia stabilita dal Cielo, diventa in Mozi la "volontà celeste" che viene rispettata pena le sofferenze inferte da schiere di demoni.
\section{Zhuangzi (IV secolo a.C.)}
Nel nuovo contesto storico del periodo degli Stati Combattenti le riflessioni filosofiche partono da un enfasi della natura conflittuale e violenta della realtà che li circonda, e cercano di trovare delle vie (Dao) attive per stabilire l'ordine, come fanno i confuciani con la ricerca del Ren. La corrente taoista invece opta per il "non agire" e mettersi all'ascolto del Dao (metafora tipica del Dao come una musica). I primi due maestri taoisti sono Laozi e Zhuangzi, tradizionalmente il primo personaggio è considerato contemporaneo di Confucio ed il secondo successivo, del periodo degli Stati Combattenti, tuttavia un'attenta analisi dei testi (attenzione a distinguere tra l'autore a cui si affibbia il testo ed il testo in sè) non riesce a stabilire con certezza quale dei due maestri sia venuto prima. \'E importante tenere a mente che l'accostamento di questi due nomi sotto la scuola del taoismo è un fatto risalente alla prime età imperiale, e dunque posteriore alle vite di questi maestri. I testi di questi due maestri portano il loro stesso nome.
\subsection{Opera e personaggio}
Il testo è scritto in una prosa esuberante, di alta qualità letteraria e poetica, ha una forte presenza della persona dell'autore. Nonostante il tema che si ripete nel testo è l'esaltazione di una vita di ritiro e disimpegnata, all'interno convivono diverse correnti di pensiero che appartengono non solo a quello che è tradizionalmente riconosciuto come autore ma anche a pensatori di epoche successive (solo i primi 7 di 33 capitoli sono certamente di Zhuangzi). Proviene da un contesto culturale più raffinato ed esuberante di Confucio, che era più legato alla cultura ritualistica tradizionale. Di lui si sa poco: ha avuto un incarico politico minore che ha in seguito abbandonato per ritirarsi dal mondo, è ritenuto una figura eccentrica. 
\subsection{Il rifiuto della ragione e il relativismo linguistico} 
La riflessione taoista parte dal presupposto che il Dao è il corso naturale delle cose che bisogna lasciar andare, l'uomo tende a distaccarvisi, convinto della sua posizione centrale nell'universo, nel suo attivismo e soprattutto nell'uso del linguaggio. Zhuangzi deride di continuo la ragione discorsiva, fa spesso uso di giochi linguistici, paradossi ed ironia per uscire dalla logica ordinaria, evidenziando il relativismo del linguaggio. In questo si schiera contro la pratica di disquisizioni argomentative e logiche, diffusa nel periodo degli Stati Combattenti (temi più vicini allo stile occidentale: relatività dello spazio e tempo, similitudine e differenza). Mentre però i contemporanei tendevano a trovare nel linguaggio un riferimento affidabile contro il relativismo di questi concetti, Zhuangzi scredita anche il linguaggio e la ragione discorsiva in generale. L'apprendimento allontana dal Dao, che è una realtà talmente altra da non essere avvicinabile razionalmente. Il linguaggio non può dirci nulla sulla vera natura delle cose perché nominando suddivide artificialmente la realtà, intrinsecamente unitaria e senza delimitazioni ed opposizioni. Inoltre le distinzioni operate dal linguaggio non hanno neanche validità interpersonale certa, in quanto possono essere interpretate soggettivamente in modo arbitrario, non essendo legate alla realtà. Si deride Confucio come colui che ha affermato di conoscere quando questo è impossibile. La soluzione è quella di non affermare nulla. Ma si ferma qui il pensiero esposto nel Zhuangzi? Nei capitoli di non certa attribuzione si individua sia la tendenza di distruggere il linguaggio e la ragione in favore dell'assurdo sia quella di dimenticare e superare il linguaggio in vista di qualcos'altro. Il saggio dunque si rende conto dell'artificialità del linguaggio e ne fa uso consapevole che vi sia qualcosa che va oltre.\\
I taoisti, data la limitatezza del linguaggio, usano tre \textbf{metafore} per descrivere la condizione dell'ideale umano:
\begin{itemize}
	\item Specchio: \textbf{spontaneità}, lo specchio riflette senza distorcere, che sul piano umano equivale al seguire il dao senza pregiudizi, vivendo il momento. 
	\item Acqua: \textbf{cedevolezza}
	\item Eco: \textbf{immediata}
\end{itemize}
\subsection{La conoscenza utile}
Come un corso d'acqua scorre naturalmente seguendo i rilievi senza modificarli, l'uomo saggio solo mediante l'inattività può abbandonarsi alla natura delle cose e seguire il Dao. L'unico tipo di conoscenza utile a tale scopo non è quella intellettuale ma quella che si acquisisce nel tempo, come quella dell'artigiano, legata alla naturalezza delle azioni più che al ragionamento (esempio del cuoco che squarta un bue con maestria).Questa concezione della conoscenza è stereotipicamente legata all'immagine popolare della Cina nel lottatore d'arti marziali che si allena duramente per raggiungere una perfetta e consapevole naturalezza nella sua arte. La spontaneità esaltata dal taoismo non è legata alla libertà del genio come quella romantica ma al contrario è necessaria in quanto segue l'unica e perfetta natura delle cose, non c'è posto per l'io ma anzi si deve trascendere la coscienza calcolatrice (oblio di sè) che è solamente limitante. Zhuangzi non è un irrazionale ma un antirazionalista, poiché eliminata la ragione non si resta con nulla, la realtà non è prodotto dell'immaginazione ma resta la natura unitaria ed immediata. A differenza del confucianesimo che crede nell'apprendimento delle cose del mondo esterno, il taoista tende a pensare che una volta in sintonia con il dao è possibile conoscere il mondo dal chiuso di una stanza.\\
Sogno della farfalla di Zhuangzi: sogna una farfalla e non sa se è la farfalla che sogna di essere Zhuangzi o Zhuangzi di essere una farfalla. Questa storia non vuole evidenziare l'inesistenza della realtà ma è ancora una volta un esempio di come la conoscibilità della realtà sia limitata, di come essenzialmente le cose siano indistinte e l'individuazione illusoria. L'affibbiare i nomi per distinguere crea ordine illusorio.
\subsection{L'ideale}
L'ideale che secondo Zhuangzi l'uomo dovrebbe perseguire è il liberarsi sia dalla propria natura umana, che impedisce all'uomo di raccordarsi con il Cielo da cui proviene, sia dalle emozioni e dai sentimenti, il tutto finalizzato a fondersi col Dao, l'ordine naturale delle cose. Si noti l'opposizione al confucianesimo che invece si fonda sul ricercare e affermare la propria umanità. Questo uomo ideale è incarnato dal Santo, in cui la distinzione fra Uomo e Cielo viene a mancare, in un'estasi divina in cui il corpo diventa mero involucro. Per far ciò non solo lo spirito deve essere allenato ma anche il corpo, da ciò l'usanza taoista di praticare ginnastica, meditazione e controllo della respirazione. La fusione con il Dao non è da intendersi come un'estasi in cui ci si fonde con il tutto ma come un distacco che si ottiene con una lunga preparazione per cogliere più profondamente la realtà e non per negarla. La morte stessa perde la negatività in quanto un semplice passaggio delle cose nel contesto del fluire del Dao.\\ 
Infine in Zhuangzi si trova più volte, lasciata in sospeso, la domanda sull'esistenza di una divinità creatrice del tutto oltre che regolatrice.

\section{Mencio IV secolo a.C.}
Contemporaneo di Zhuangzi (anche se probabilmente non lo ha mai incontrato), Mencio vive nello stesso clima di crisi intellettuale ed istituzionale, il discorso si sposta intorno alla razionalità e alla giustificazione teorica delle proprie interrogazioni (si pensi al discorso sul linguaggio di Zhuangzi). Il suo pensiero è strettamente legato al periodo in quanto il confucianesimo di Mencio è applicato più sistematicamente per offrire risposte ai disordini dell'epoca: da cosa dipende il disordine? Dall'uomo, cioè dalla sua natura (\textbf{xing} = "natura umana). \'E discepolo di un successore di Confucio ed è considerato il suo erede spirituale. Della sua biografia si sa poco: peregrinò di paese in paese verso i sessanta anni e sente il mandato di una "missione celeste", il suo ideale è quello dell'"uomo di valore" che deve seguire la Via tracciata dagli antichi (tipicamente Confuciano). Le parole associate a Mencio sono innatismo, idealismo ed ottimismo. La sua riflessione parte dall'esperienza: osserva l'armonia della natrua ed attribuisce all'uomo una natura positiva in quanto parte di questa armonia.
\subsection{Il principe e lo shi}
Si pensa che la sua peregrinazione fosse destinata alla ricerca di un nuovo re saggio, che credeva dovesse comparire ogni 500 anni. Nel periodo degli Stati Combattenti il letterato ricopriva la figura di consigliere di stato itinerante che poteva scegliere il principe da servire o ritirarsi dalla vita pubblica. Caso pressoché unico nella storia cinese, l'intellettuale godeva quindi di una discreta libertà dal potere politico. Presto l'intellettuale sarebbe stato integrato nell'apparato statale in una figura propriamente politica, lo shi, che viveva tra vita activa e vita contemplativa. Lo shi è una figura già presente in Confucio, ma Mencio per la prima volta distingue nettamente la virtù morale dell'"uomo di valore" (nobiltà del Cielo), da quella politica del principe (nobiltà umana).\\
L'opera di Mencio è il Mengzi, nel capitolo V tratta del rapporto fra principe e shi, quest'ultimo può essere visto come maestro, amico o servitore, ma la relazione d'amicizia è difficilmente sostenibile. Ne segue che lo shi rispetto al principe è maestro o ministro,  nel primo caso è superiore al principe dal punto di vista etico, nel secondo gli è inferiore dal punto di vista politico. Il rapporto è complesso in quanto entrambi devono legittimarsi a vicenda in quanto il principe deve legittimarsi moralmente e lo shi deve aver riconosciuto il suo status privilegiato dall'autorità politica. Il risultato di questo rapporto in Mencio si traduce nella declinazione più politica del Dao.  
\subsection{Il Mengzi: confucianesimo e politica}
Mentre nei Dialoghi di Confucio la parola si proferisce senza sforzo, in Mencio si sente la necessità di raffinare il discorso per fronteggiare gli attacchi delle "cento scuole" del periodo degli Stati Combattenti. Avviene una tendenza analoga a quella della sofistica greca in cui l'affabulazione e la spregiudicatezza (o al massimo l'amoralità) diventano preminenti rispetto alla morale. In questo contesto Mencio risulta isolato ed è percepito come un idealista: si propone di difendere la dottrina di Confucio, raffinandola in modo da rivaleggiare con le idee del suo tempo. In questo modo Mencio precisa temi che in Confucio erano solamente accennati ed in generale ne fa una rivisitazione immettendo elementi di altre correnti. \'E presente un tono polemico e difensivo, volto a convincere, assente nei Dialoghi; anche la forma letteraria passa da frammenti di dialoghi ad un'opera discorsiva di più alto valore filosofico (probabilmente Mencio stesso contribuì direttamente al Mengzi).\\
Il messaggio centrale è quello etico-politico: il miglior modo di governare è rendere operante il ren, ma questa riproposizione della scommessa sull'uomo di Confucio suona in questo nuovo contesto molto debolmente. Mencio risponde che solo con il ren (a differenza degli altri ideali pragmatici e amorali) è possibile governare con il consenso del popolo e dunque garantire coesione e stabilità. I sovrani potevano abbracciare questo pensiero per motivi sia pratici che morali: da una parte, essendo che all'epoca la potenza di uno stato era determinata in gran parte dal numero di abitanti, uno stato umano poteva attirare persone, d'altra parte i sovrani necessitavano di una legittimazione morale dal popolo in quanto la spontanea adesione di questo al principe era l'espressione di star seguendo il mandato del Cielo. La giustificazione da parte del Cielo del potere regale era ben presente nel pensiero cinese e Mencio la radicalizza, in particolare con l'idea che il popolo è legittimato a sovvertire il potere se questo non si mostra legittimato dal Cielo (si accettata la possibilità del \textbf{regicidio}, a piazza Tienanmen i ragazzi citavano Mencio). L'idealismo di Mencio non è statico ma, come nel concetto di zhong, dinamico. La morale prevale sulla politica in Mencio, questa idea si radicherà profondamente nel senso cinese delle istituzioni, fino ai giorni nostri. Non bisogna vedere Mencio come un democratico, crede nella meritocrazia ma non esce dalla gerarchia, il ren non porta all'eliminazione della gerarchia ma anzi la giustifica moralmente (se tratti i tuoi sottoposti con umanità legittimi la tua posizione). La suddivisione gerarchica della società è giustificata dal fatto che non è possibile soddisfare i propri bisogni autonomamente ed è dunque naturale che vi siano quelli che fanno lavori manuali che sono governati da quelli che fanno lavori intellettuali, e che hanno sostentamento grazie ai primi, gli shi si presentano come tecnici della politica (divisione classica in Cina tra lavoro intellettuale e manuale). In questo si nota la differenza da Confucio che rifiuta ogni specializzazione.  
\subsection{Moralità, natura e xin}
Il concetto di umanità (ren), solamente abbozzato in Confucio, è rielaborato da Mencio che afferma con vigore la natura benevola dell'uomo; in COnfucio la natura umana è perfettibile ma in sè può essere malvagia o benevola, in Mencio è essenzialmente benevola. Quando il pensiero cinese si fa più sistematico, tutte le correnti degli Stati Combattenti si interrogano sul rapporto Uomo-Cielo, e della natura umana (cosa ci è dato dal Cielo?). Se Zhuangzi propendeva dall'estremo del Cielo (l'uomo deve eliminare i suoi tratti umani per unirsi al Cielo), Mencio sostiene che il senso di moralità viene dal Cielo, dunque seguirlo equivale a seguire l'ordine di natura. Si noti come sostenere che la moralità sia naturale, ed in particolare l'empatia e il rifiuto del vedere e fare soffrire, equivale ad affermare la naturale bontà dell'uomo. In questo modo Mencio integra il non agire Taoista nel confucianesimo, poiché se bisogna non agire per seguire la natura, ma la moralità è naturale, allora agire moralmente equivale al non agire taoista. Mencio parla di \textbf{4 germogli} naturalmente presenti nell'uomo e che possono esser coltivati da tutti:\textbf{ benevolenza, rettitudine, osservanza rituale, saggezza}. Un fatto interessante è che in questo modo la moralità è legata all'istinto e non alla ragione, se la si considera come esterna all'uomo e si cerca di accrescerla razionalmente si fa come il contadino di Song che tira le piante per aiutarle a crescere, e le fa appassire. Interessante notare come da un lato per Mencio il qi irradia e permette di germogliare la moralità umana mentre per Zhuangzi la moralità è massimamente contraria al naturale fluire del qi.\\
Cosa distingue l'uomo dall'animale? A priori nulla, non vi è qualcosa dato dal Cielo all'uomo che lo distingua se non un "nonnulla": lo \textbf{xin} (cuore/animo), di per se molto poco sviluppato. Questo termine indica la sensibilità umana, intesa anche come pensare ciò che si sente; è interessante notare come nella Cina classica il cuore sia sede sia dei sentimenti che dell'intelletto, a differenza della dicotomia europea testa-cuore (il cinese non pensa alla dicotomia mente corpo perché tutto è qi). Se si accresce questo "nonnulla" e ci si differenzia dagli animali coltivando la propria umanità si diventa "uomini di valore". Si riafferma dunque la convinzione nella possibilità di continuo miglioramento dell'uomo "Chiunque può diventare Yao o Shun", con più forza che in Confucio, che poneva i Santi del passato come limiti irraggiungibili.\\
Ma allora che ne è del male per Mencio? Un uomo malvagio, per il concetto di umanità di Mencio, non è un uomo poiché non ha pienamente sviluppato la sua natura benevola o non ne ha preso coscienza. A differenza del racconto biblico della caduta dell'uomo dall'eden e dell'intrinseco germe di malvagità in esso, la malvagità nell'ottica di Mencio non è irreversibile in quanto chiunque in qualunque momento può cominciare a coltivare i propri germi del bene. L'egoismo e l'individualismo sono visti interamente in modo negativo e non è posta enfasi sulla libertà della scelta fra bene e male in quanto la Via è unica. Confucio aveva introdotto in modo indelebile un'ottimismo nei confronti dell'uomo che non permette una drammatizzazione della natura umana, come avviene nella Bibbia o nel Buddismo. La questione del male nel pensiero Cinese non sarà seriamente affrontata fino al Buddhismo. Manca il libero arbitrio nel pensiero di Mencio, ma vi è il senso della responsabilità nell'assecondare l'onere di coltivare i propri germogli e diventare umani (la responsabilità di essere umani).\\
In Mencio si parla di \textbf{fisiologia morale}, cioè l'idea secondo cui seguire la natura è in sè morale e porta al bene, sia dello spirito che del corpo ("comportarsi bene fa bene alla salute"). Si noti come in questa idea vi siano gli echi della corrispondenza fra Macrocosmo e Microcosmo tipica del pensiero cinese secondo cui seguire la propria natura, e dunque stabilire un'armonia individuale, contribuisce all'armonia del tutto e che l'ordine cosmico è riprodotto in piccolo nell'ordine individuale. Questa idea tenta di mettere insieme la moralità confuciana incentrata sull'azione umana con quella taoista che ha come massimo ideale quello di seguire il dao della natura, Mencio sente la necessità di rispondere ai taoisti.\\
Il male è ignoranza in Mencio ed in gran parte del pensiero cinese.
\section{Laozi (V secolo a.C.)}
Il nome vuol dire "vecchio maestro" (nato vecchio a seguito di 62 anni di gestazione), si dice avesse lunghe orecchie, tratto degli immortali.  Tradizionalmente si ritiene contemporaneo di Confucio, apre la Via taoista mentre Zhuangzi sarebbe un successore. Dai testi tuttavia si deduce (anche alla luce dei suoi interessi, tipici del periodo degli Stati Combattenti) che è probabile sia vissuto nel II-III secolo a.C. Il suo testo omonimo, anche detto "Classico della Via e della sua Virtù" (Daodejing) ha uno stile unico, fatto di brevi e oscuri versi in rima (esistono svariate interpretazioni e traduzioni). Un fatto interessante è che questo testo, considerato fondamentale per il taoismo, durante la sistematizzazione delle opere di epoca Han, non è inserito nel canone taoista ma è visto come scritto politico. Solo in un secondo tempo si sottolineerà l'aspetto astratto del taoismo, talvolta eccedendo nell'estremo opposto e scordando il legame con la vita vissuta, i riti propri del taoismo.
\subsection{Il non agire}
Un tema che porta a datare il Laozi all'epoca degli Stati Combattenti è quello del problema di uscire dal circolo vizioso della violenza. La risposta del Laozi è il "non agire", sottolineando come le Vie degli altri pensatori abbiano provocato il declino del Dao. Si parte col constatare che la forza si ritorce sempre contro se stessa, bisogna allora essere come l'acqua che umile, cedevole, trasparente, riesce però ad intaccare i materiali più resistenti, il debole trionfa sul forte in modo pacato (la si metta in rapporto alla metafora dell'acqua del Mengzi). Questo concetto si può ritrovare nelle arti marziali cinesi: judo vuol dire "la via del molle" ed è basato su prese che, a seconda dell'agire nemico, volgono indietro la sua forza. Anche da questo esempio sportivo si intuisce come il non agire non sia il non far nulla ma astenersi dall'azione aggressiva ed intenzionale, agire unicamente secondo Dao.
\subsection{Il paradosso e l'amoralità}
Una caratteristica del Laozi è la forma paradossale che consiste nel contraddire comuni abitudini del pensiero, ad esempio preferendo il debole al forte, il femminile al maschile, l'ignoranza alla conoscenza (in opposizione al confucianesimo). Si ponga attenzione al verbo "preferire": nel pensiero cinese non vi sono mai forti opposizioni e ciò che non è preferito non viene comunque eliminato (no terzo escluso ma organicità). Il paradosso più radicale è il preferire il nulla al qualcosa, il vuoto al pieno. Il Laozi è più radicale del Zhuangzi, poichè quest'ultimo si limitava a deridere linguaggio e ragione, mentre il primo esalta l'ignoranza. Implicita in questa prospettiva sta l'idea che ogni cosa, anche quelle forti e superiori sono state in origine deboli ed inferiori e prima o poi faranno ritorno alla loro origine. Il paradosso si scioglie osservando che il saggio, ponendosi in una posizione inferiore, debole, fa si che gli altri arrivino certamente prima o poi al suo stesso punto, convergeranno ad esso. Da ciò deriva la tipica tolleranza taoista, che non si impone ma aspetta che le differenze si acquietino. Questo tipo di tolleranza non ha alcuna connotazione morale come nel cristianesimo, ma muove dalla volontà di non andare controcorrente ma di seguire il Dao. A differenza che in Mencio il Dao nel Laozi è intrinsecamente amorale e privo di ren. I taoisti sostituiscono il Mezzo all'Origine come elemento da ricercare per seguire il Dao. 
\subsection{Teoria politica}
Il non agire porta all'ideale di un sovrano che deve imporsi attivamente il meno possibile ed evitare la violenza, per far seguire al paese il suo corso naturale. Per far ciò bisogna che nel popolo non sorgano contrasti, questi scaturiscono dall'agire, cioè dal dar valore alle cose, dalla meritocrazia... La soluzione è dunque adottare una vita frugale, che però può sfociare in una visione abbrutente dello stato, in cui regna l'ignoranza, in cui sono garantiti solo i beni necessari e non si da importanza al progresso e allo sviluppo tecnologico. \'E un luogo comune che il taoismo sia una forma di conoscenza personale, e può dunque stranire la presenza di una teoria politica, ad esempio assente in Zhuangzi; al contrario nel Laozi l'ambito politico è quello in cui si applica più marcatamente l'ideale del non agire. Il Laozi, far le varie interpretazioni, può esser letto come testo politico. Il non agire è un modo per tornare ad uno stato di naturalezza, come alla nascita, cioè all'Origine perduta. dal punto di vista politico ciò si configura nella speranza di tornare ad una società anteriore alla formazione dello stato, fatta di piccole comunità autarchiche, nella convinzione che l'uomo sia per sua natura privo d'aggressività (da un punto di vista diverso torna un certo ottimismo nei confronti dell'uomo, come già visto nei confuciani). 
\subsection{Il Dao}
Si distingue il Dao, l'unico naturale fluire della realtà con i dao, cioè le vie naturali delle singole molteplicità; queste ultime comunicabili, la prima no. Ogni opposizione è risolta nel Dao, che è Uno; la limitatezza della nostra presa sul mondo non è dovuta ai nostri sensi ma, come in Zhuangzi, al linguaggio che inscatola la realtà in distinzioni fittizie. Sempre dal linguaggio proviene lo slancio all'azione, contrapposto al Dao che è quiete. Il Dao però non è da pensare come trascendente la realtà, quanto più come elemento generativo e della realtà in modo dinamico, mentre questa si evolve. Una parola-chiave per comprendere il Dao è "ritorno", in quanto risalendo dallo stato umano a quello originario di bambino si va ancora avanti a quello di "c'è" per poi giungere al culmine, "l'ancora non c'è", che è la fusione col Dao. In questo senso il Dao si comprende negativamente, per sottrazione, in aperta contrapposizione alla concezione confuciana, secondo cui il Dao si raggiunge con l'apprendimento e la tensione verso la moralità. Il ritorno al Dao si configura come un'esperienza mistica, che è l'unica alternativa, prima del buddhismo, alla scommessa confuciana sull'uomo.   
\section{Il taoismo religioso}
Si scoprono nuovi testi Han all'indomani della caduta dell'impero, che nel secondo secolo d.C. entra in crisi; si sgretola non solo il potere politico ma tutta una visione della realtà. Cominciò una crisi del potere politico che si divise in tre regni in competizione per il potere, che si sarebbe riunificato nel 589 d.C. Gli intellettuali si rivolgono ad una filosofia meno impegnata nel sociale ed ottimista nei confronti dell'umanità: il taoismo ed il suo misticismo (forte analogia con crisi dell'impero romano e impero alessandrino). Si forma la prima chiesa taoista, nel 142 d.C. Zhang Daoling  ha una visione di Laozi, visto come manifestazione (non è il Tao stesso perché questo è impersonale) del Tao, che lo spinge a stringere un patto fra mondo divino ed umano (ricorda il mandato celeste confuciano), non per far diventare Zhang Daoling imperatore ma per creare un regno teocratico utile per creare il terreno per far tornare un nuovo imperatore legittimo. A partire dal 142 comincia l'esperienza dello stato teocratico di Zhang Daoling. Questa costituisce la prima esperienza propriamente religiosa in Cina, che sarà affiancata da un certo tipo di Buddhismo religioso (taoismo religioso è la religione non ufficiale in Cina). Zhang diventò il primo Maestro Celeste, ed iniziò a diffondere la sua setta. Primo di una lunga serie di Maestri che arriva fino ad oggi, i Maestri Celesti si succedono per eredità e sono i massimi detentori del sapere taoista, sotto ai quali vi sono dei sacerdoti che formano una vera e propria burocrazia. Non è uno scimmiottare della struttura burocratica degli Han ma è un riprendere la discendenza religiosa della politica, tipica della cultura cinese (prima la religione poi la politica). I maestri celesti sanno dialogare con le forze della natura proprie del taoismo religioso, che vede gli spiriti come potenzialità, che chi le conosce può sfruttare a suo favore. La cosmologia taoista è tripartita: Cielo-Terra-Inferi. Nel 1949 con l'innesto del Maoismo il maestro celeste scappa a Taiwan e vi impianta questa religione. \\
Nel 1200 nasce un altro filone di tradizione religiosa taoista di tipo monastico (simil-buddhista, anche per la competizione fra i due pensieri), si chiama Zhuang Zen. Ci sarà un tira e molla con il potere imperiale, che talvolta la accetta nonostante sia uno "stato nello stato" (che va contro l'idea di totale adesione che vorrebbe l'impero) ma altre volte ci saranno addirittura persecuzioni. Le frizioni fra potere e religione non hanno mai basi ideologiche, in oriente il credo è un fatto personale, i problemi sorgono a livello politico. Il taoismo monastico è accettato perché mediante il monastero è più facile controllare il potere religioso. 
\section{Xunzi (III secolo a.C.)}
\'E parte della seconda ondata di pensatori del periodo degli stati combattenti, presenta ancor più marcatamente di Mencio un tono polemico e il suo pensiero vive in relazione alle idee che critica, nell'ambito delle 100 scuole. Xunzi vive durante la maturità di Mencio, manualisticamente si è soliti individuare in questi pensatori due diramazioni opposte del confucianesimo, la prima "idealistica", estremizzando l'enfasi nella scommessa sull'uomo, e la seconda "realistica", ponendo l'accento sul rigorismo. Durante la rivoluzione culturale (1966-1976) il dibattito fra le correnti di pensiero del periodo degli stati combattenti si sarebbe ridotto a confucianesimo contro legismo (di cui si parlerà in seguito) e Xunzi sarebbe stato visto come un legista, nonostante per millenni sia stato considerato confuciano. Il Xunzi è formato da 32 capitoli in cui ciascuno forma un trattato su uno specifico argomento, l'unica opera del pensiero cinese antico ad essere continuativa e strutturata e segna la tendenza sempre più marcata ad un pensiero discorsivo e razionale. Anch'egli studia negli stati centrali con influssi culturali analoghi a quelli di Confucio e Mencio. Studia nell'accademia Jixia (dove si era formato anche Mencio), voluta da un re per render forte il paese unificandolo sotto la cultura confuciana, si offriva ogni agio ai pensatori, che accorsero formando una grande scuola che favorì l'ascesa della classe shi. Gli intellettuali non hanno tuttavia ruoli politici, ma avendo la funzione di legittimare il potere politico, si scontrano per ottenere l'egemonia ed imporre allo stato il proprio Dao. Il Dao è dunque percepito come principio d'ordine da applicare allo stato in contrapposizione al caos del periodo degli Stati Combattenti. Xunzi diventa primo ministro ed è testimone della caduta della dinastia Zhou ma non vive abbastanza per vedere l'unificazione sotto l'impero. Un tratto del pensiero di Xunzi è la contemplazione della violenza, affiancata all'umanità, come strumento politico, ponendosi a metà fra Mencio e i legisti.
\subsection{L'uomo} 
Data la caduta della dinastia Zhou, un tema centrale della seconda parte del periodo degli Stati Combattenti è la messa in discussione del ruolo del Cielo nell'assicurare l'armonia del mondo umano, il rapporto fra questi due poli della realtà deve dunque essere ripensato (capitolo 17 del Xunzi interamente dedicato al Cielo, unicum nella cultura cinese preimperiale). In Xunzi l'uomo non è più interamente dipendente dal Cielo, l'ambito etico-politico è più separato che nei pensatori passati da quello celeste(ma mai opposto, non ci sono opposizioni nel pensiero cinese). L'opera del Cielo continua a garantire l'armonia della natura sulla Terra (ma non dell'uomo) ma si sottolinea il suo carattere di inconoscibilità, il saggio non ci prova neanche. Si presenta per la prima volta una visione triadica Cielo-Terra-Uomo, in cui quest'ultimo assurge a potenza cosmologica, separata dalle altre ed il cui ordine non è garantito da queste. L'uomo deve impegnarsi per produrre il suo proprio ordine e non ha senso abbandonarsi a congetture su ciò che trascende l'uomo in quanto inconoscibile. Vi è un razionalismo che demistifica le capacità sovrannaturali riti e la superstizione a favore del buon senso pratico, questo orientamento porta anche ad un disinteresse nella ricerca scientifica a cui si preferisce la politica (di nuovo si contrappongono gli omofoni LI come ordine e li come senso rituale).\\
Nell'opera di Xunzi il rito non è illuministicamente negato ma svolge un ruolo fondamentale, come in tutto il confucianesimo, di mezzo sovrannaturale per coltivare la saggezza umana in modo da creare ordine, del rito si critica solamente la pretesa di agire direttamente sul Cielo e la natura ma non la possibilità di migliorare l'uomo. Vi è dunque un primo paradosso in Xunzi: da un lato il rapporto tra Cielo e Uomo, che pur essendo separati sono comunque connessi dal rito che coltiva l'uomo mediante un rapporto col Cielo. Un'altro fatto unico ed interessante è l'idea per cui la natura umana è intrinsecamente malvagia (dichiarazione di guerra a Mencio), esseno costituita di brama per il guadagno, avidità e aggressività, soddisfacimento animale degli istinti. \'E dunque necessaria la cultura e i riti per poter artificialmente emendare questa natura (raddrizzare il ramo storto), anche mediante l'uso della forza che viene quindi in parte legittimata (spirito legista di Xunzi). La malvagità dell'uomo però si riduce al suo essere innanzitutto animale, con necessità primarie amorali, che possono però essere gestite con la cultura e i riti. La natura umana proviene dal Cielo che però è amorale (a differenza del Cielo in Mencio che fonda la moralità). L'uomo si distingue dagli animali solo in virtù della sua intelligenza, fonte della moralità, che è vista essenzialmente come capacità di discernimento (che era uno dei quattro germogli di Mencio) ed è frutto dello xin (cuore/animo) che comprende se una pulsione naturale sia morale o meno. A differenza che in Mencio, ed avvicinandosi a Mozi, il cuore/animo è una facoltà che valuta i pro e i contro, tuttavia Xunzi si differenzia da Mozi avvicinandosi al confucianesimo nell'importanza che attribuisce ai riti: anche il mangiare, fatto istintivo, può essere elevato dall'intelligenza ad un atto etico trasformandolo in rito (attribuendo significato ai propri atti istintivi).\\
Infine, questa prospettiva risponde alla tendenza taoista dell'abbandono dell'uomo al Cielo, al naturale intrinsecamente giusto, riaffermando la facoltà umana di produrre ordine indipendente dal Cielo. Vi è in Xunzi un'importanza centrale della cultura, intesa come prodotto umano artificiale. Quello che era il ruolo dell'umanità in Confucio e Mencio è ora affidato ai riti e alla cultura. Ma da dove nascono i riti? Dall'uomo, in particolare dai Santi del passato, che conoscendo l'utile umano hanno capito come ottenerlo: istituendo i riti come \textbf{convenzioni}, organizzando l'uomo in società e ripartendo equamente le risorse (visione che riecheggia il contrattualismo occidentale). Interessante notare come il principio di ripartizione sia visto come inscritto nel senso rituale.   
\subsection{I nomi}
In Xunzi ritorna il tema confuciano della rettificazione dei nomi, che nel periodo degli Stati Combattenti assume torna centrale per il nuovo interesse verso la logica e l'argomentazione discorsiva. L'attribuire nomi è concepito come tracciare distinzioni nella realtà. \'E sorprendente la radicata idea che vi sia una corrispondenza fra la molteplicità della realtà percepita dall'uomo e le parole con cui ci si riferisce ad essa, si critica infatti l'eccessivo proliferare delle parole, che intacca questa corrispondenza. La rettificazione dei nomi è proprio la re-istituzione della corrispondenza perduta. In Xunzi l'importanza della rettificazione è legata all'importanza che la conoscenza umana ha nel regolare la realtà, nel distinguere il morale dall'immorale: senza una chiarezza data dall'adozione di un linguaggio appropriato per descrivere la natura è impossibile per l'uomo istituire l'ordine e trovare la moralità. \textbf{Per Xunzi i nomi sono convenzionali come i riti}, non descrivono la realtà in modo essenziale ma sono comunque utili (l'interesse di penetrare nella realtà non è mai forte nel pensiero cinese) nella dimensione umana della realtà. 
\subsection{I riti}
 Un tratto interessante del pensiero di Xunzi è la sua complessità, che scaturisce dalla volontà di trattare di tutte le questioni e confrontarsi con pensatori di altre scuole come della sua stessa. Ciò che mantiene la coesione in questo pensiero è il concetto di rito, che risolve le contrapposizioni Cielo-Uomo, la ricerca del Dao, la moralità, la natura umana. I fin dei conti, per quanto la concezione di uomo in Xunzi sia lontana da quella confuciana, il discepolo non fa altro che riaffermare in termini diversi la fiducia confuciana nell'uomo, che ha addirittura un ruolo tanto importante che quello del Cielo. In questo Xunzi beneficia della possibilità di confrontarsi con un passato ed una contemporaneità ricca di prospettive; in un passo critica la maggior parte dei suoi predecessori in quanto "obnubilati" da una singola idea, che fa perdere una visione d'insieme. Il confucianesimo misto al legismo inaugurato da Xunzi si imporrà con l'avvento dell'impero. 
\section{I legisti}
Inizialmente non si propone come riflessione filosofica ma come insieme di pratiche relative alla politica, le maggiori opere sono intitolate a ministri, ad esempio il Guanzi, un famoso ministro a cui Confucio stesso fa gli onori senza approvarne i metodi. Tutte le correnti antiche del pensiero cinese si occupano di politica ma nessuna quanto il legismo, incentrato sul preservare e rafforzare il potere dello stato. La sistematizzazione di pratiche già esistenti col tempo si è tradotta in una più complessa visione del mondo. Sono la prima corrente di pensiero che fa tabula rasa della tradizione, considerando l'uomo e la società non per come dovrebbe essere ma per come è (nell'epoca degli Stati Combattenti c'è un allontanamento generale dalla tradizione). Alla tradizione si sostituisce un'analisi storica volta a capire come vivere al meglio il proprio tempo, adattandosi ai cambiamenti. propongono un'analisi quasi malthusiana delle origini dell'umanità : inizialmente generati dal Cielo e Terra gli uomini erano legati solo da parentela ed egoismo ma crescendo la popolazione ciò portò al disordine fin quando gli uomini di valore non istituirono un'ordine basato sull'altruismo, l'umanità e la promozione dei più capaci. Questa condizione però porta ancora al disordine dato dalla volontà di superarsi l'un l'altro. Si istituirono allora le divisioni, le leggi e i divieti, nacque l'istituzione del sovrano che regna per diritto di sangue al posto del modello della promozione dei più capaci, che altrimenti creerebbero disordine superandosi l'un l'altro. Si stabilisce così l'ordine. La progressione della storia non ha carattere benevolo come per i confuciani o di decadenza come per i taoisti, è semplicemente presa come un dato. 
\subsection{La legge}
Appare chiaro che i legisti sostituiscono la legge ai principi morali del ren ed allo spirito rituale, come mezzi per trovare l'ordine. Se le leggi sono giuste, non c'è bisogno di introdurre la morale e la soggettività poiché queste, come la bilancia che indica il peso degli oggetti senza alcun intervento umano, indicano ciò che è oggettivamente giusto, la legge basta a se stessa. Risale al periodo delle primavere e degli Autunni la prima iscrizione di leggi su tavolette di bronzo che ebbero l'effetto di avversare la struttura sociale feudale basata sulle relazioni rituali. La legge rende più uguali gli uomini, mentre le leggi rituali erano interiorizzate dalla società, che le rispettava naturalmente, le leggi si impongono attivamente, dall'esterno, oggettivamente. La legge è intesa essenzialmente in senso penale: essa stabilisce le ricompense e i castighi che sono le uniche influenze che possono agire sulla natura umana, che è vista con molta minore considerazione rispetto ai pensatori precedenti. \\
\subsection{La politica: legge posizione di forza e tecniche di controllo}
I legisti dunque separano politica e morale e si preoccupano di come costituire una società stabile indipendentemente dalle qualità morali del sovrano. Ciò è possibile solo grazie alla "posizione di forza" che deve mantenere un sovrano e che ne esclude il valore morale personale, in generale la centralità della persona del sovrano decade in favore della forza dell'apparato statale. Questo rispecchia un passaggio da ideali aristocratici e ritualistici ad una mentalità più volta alle istituzioni e alla burocrazia. Oltre alla legge dunque diventa importante la tecnica politica con cui il sovrano fa rispettare le leggi e con cui sceglie l'apparato governativo (seguendo il principio della "promozione dei più capaci"). la scelta dei più capaci però non è arbitraria da parte del sovrano ma oggettiva (si introducono per la prima volta test per i funzionari), il sovrano ha il ruolo di verificare costantemente la corrispondenza tra pratica effettiva e ruolo dei funzionari. In discorso sulla rettificazione dei nomi assume la connotazione politica della corrispondenza fra ruolo, associato al nome che lo designa, e realtà. I legisti hanno il merito d'aver capito che non si governa con le buone intenzioni ma con solide istituzioni. Lo stato legista è totalitario in quanto, essendo il suo buon funzionamento assicurato dalla razionalità con cui è costituito, non c'è spazio per deviazioni. \\
\subsection{Han Feizi: la rilettura legista del Laozi}
Il testo principale del legismo è lo Han Feizi, il suo autore è l'unico del pensiero antico cinese a far parte della nobiltà e a non essere uno shi. Han Fei ebbe come maestro Xunzi ed è l'unico legista a fondare filosoficamente il suo pensiero: egli rilegge il Laozi interpretando il Dao non in termini di natura ma di legge, che forma l'ordine dell'universo, mantenendo l'idea diffusa del pensiero cinese della continuità fra ordine umano e naturale. Il Dao/legge, come nel taoismo, è amorale ed indifferente agli uomini e non ha bisogno dell'intervento umano soggettivo: la legge è oggettiva e rispettandola si ottiene l'armonia, rispettare la legge equivale al non-agire in quanto si segue il naturale fluire delle cose. La differenza fra il taoismo di Laozi e di Zhuangzi che l'azione legata all'istituzione delle leggi, e del fare rispettare, anche con la violenza, sono ritenute parte del Dao e quindi annoverate come non-agire. Han Feizi, come molti altri legisti, fanno una fine curiosa perché accettando la violenza, muoiono di morte violenta (squartati o spinti al suicidio).
\section{Il pensiero cosmologico}
\subsection{L'origine della cosmologia nel proto-taoismo}
Nella Cina antica, i cui massimi rappresentanti sono Confucio e Mozi, si tentava di definire dei valori sovrapposti alla natura, scaturenti dal Cielo, come l'umanità confuciana o l'amore universale moista. Con lo sfaldarsi delle istituzioni Zhou e il conseguente declino dei riti, si ricerca all'interno della natura la fonte della saggezza, esempi di questa tendenza sono il taoismo di Zhungzi e Laozi, che si focalizzano sul naturale scorrere del Dao. In questo modo viene a crearsi un pensiero cosmologico, anche l'ordine politico è spesso fatto rientrare nella cosmologia. Si afferma, in quelle che vengono chiamate correnti proto-taoiste, l'idea di dover preservare il proprio "principio vitale", che è visto come estensione dell'ordine cosmico, anche a scapito della morale. \'E come se non bisognasse entrar a far parte della molteplicità e della mutevolezza del mondo per non distaccarsi dal Dao, si praticano dunque rigidi stili di vita. Questo è detto "materialismo di Jixia" in quanto fu probabilmente elaborato in questa accademia.    
\subsection{Qi, Yin e Yang, le Cinque fasi}
Vedremo ora alcune nozioni adottate dai proto-taoisti che fondano il pensiero cosmologico cinese:
\begin{itemize}
	\item L'origine della parola qi resta misteriosa, l'uso del carattere moderno sembra indicare il vapore che si forma sopra il riso in cottura. Il qi è il soffio vitale ma anche il principio unico della realtà che dà forma ad ogni cosa, questo porta ad una fondamentale visione unitaria dell'uomo ed il mondo circostante. \'E come se il qi si condensasse e formasse tutto ciò che esiste, quando si disperde si ha la morte, è potenzialità che si realizza quando si condensa. Dalla visione unitaria derivante dal qi segue ad esempio che salute morale e fisica coincidano per i cinesi.
	\item Lo Yin e lo Yang sono concetti moto antichi, adottati inizialmente per rappresentare le contrapposizioni naturali di giorno/notte, estate/inverno, caldo/freddo. Durante gli Stati Combattenti assumono un connotato astratto di due soffi primordiali opposti che permettono l'evoluzione dell'universo, il principio della differenza che crea. Questi non sono dunque semplicemente contrapposti ma l'uno esiste in funzione dell'altro e la loro interazione è feconda. Il qi che si muove è Yang, quando torna alla quiete diventa Yin. In questo concetto risalente ai primordi del pensiero cinese, e dunque fortemente radicato, si ravvisa la sensibilità di questa cultura per l'opposizione, che non è mai esclusione. Yin e Yang sono una categoria mentale distintamente cinese con cui si interpreta la realtà.
	\item Un'altra antica idea cinese è quella delle cinque fasi: acqua, fuoco, legno, metallo, terra. Questi non sono da pensare come i quattro elementi empedoclei, cioè come elementi costitutivi, ma più come processi che come sostanze. Questi si affiancano ai principi di Yin e Yang e, in alcuni pensatori, l'interazione di questi è descritta come una grande cosmologia. Questo si traduce in politica, dove ogni re o imperatore incarna uno di questi principi, e sceglie un colore che riporta ad esso. Ad esempio il primo imperatore, essendo caratterizzato da un governo repressivo, si caratterizza con la potenza dell'acqua, di colore nero, culmine dello Yin (mentre ad esempio gli Zhou erano il fuoco rosso, e infatti sono soppiantati dai Qin). Vediamo dunque come il potere imperiale sia scaturente e giustificato naturalisticamente e cosmologicamente. Con l'impero si ha la vera integrazione della cosmologia in ogni corrente e tipologia di pensiero. Il numero 5 ricorre nella razionalizzazione del mondo cinese, come ad esempio il numero delle stagioni, alla distinzione fondamentale delle 4 direzioni si aggiunge il centro. Ioltre ogni stagione porta con sè delle precise prescrizioni su cosa si può fare e cosa è proibito, con la convinzione profonda di dover mantenere un'armonia globale. 
\end{itemize}
\section{Il classico dei mutamenti (Ching)}
La fonte più importante riguardo il pensiero cosmologico è il Classico dei Mutamenti, che attinge da un fondo di conoscenze antichissime, ispira tutti i pensatori cinesi. \'E un insieme di frasi enigmatiche utili nelle pratiche divinatorie. Queste inizialmente si facevano interpretando ossa e gusci di tartaruga ed in seguito con gli steli di achillea e il Classico dei Mutamenti si pensa prenda parte del rito di divinazione. L'interpretazione si fa via via più astratta, basata sempre più sull'uso di numeri che rispecchia il passaggio da una mentalità religiosa ad una naturalistica in cui si crede nella connessioni fra simboli e natura più che con il divino. Nel Classico dei Mutamenti vi è una complessa tecnica di divinazione relativa ai trigrammi e alla loro composizione ed interpretazione, legata all'uso degli steli di achillea. Di fianco al testo originario vi sono le "Dieci Ali" cioè 10 commentari, che forniscono interpretazioni della divinazione fatta usando i trigrammi. L'idea di fondo è che il mutamento è in ogni cosa, è il principio fondante del cosmo, e si inscrive nella continuità armonica del Dao. Dao e mutamento sono facce della stessa medaglia: il primo però è sfuggente, è l'unità originaria, mentre il mutamento (yi) è la manifestazione visibile del Dao. Possiamo vedere la ricchezza delle molte combinazioni di esagrammi come una versione più complessa e strutturata della cosmologia legata a Yin, Yang e 5 fasi. Col passare del tempo la ricchezza dell'interpretazione degli esagrammi tenderà sempre più a formalizzarsi finendo per diventare un puro gioco meccanico.\\
Nel Classico dei Mutamenti si trova anche l'importanza che la cultura cinese attribuisce al tema del germe, dell'origine. L'inizio contiene in sè i germi di ciò in cui si evolverà. Lo studio dei piccoli indizi contenuti nell'origine è utile per poter prevedere e dominare il mutamento ed è per questo che la divinazione del Classico dei Mutamenti vi si interessa. Il naturale evolversi dall'origine al compiuto è in sè etico e giusto e tutto ciò che vi si oppone è malvagio, la relazione con il non agire del Laozi è evidente.\\
La visione che si evince da questo pensiero è di un mondo tutto connesso, in cui nulla è assoluto o indipendente. La divinazione simbolica del Classico dei Mutamenti serve a mediare fra il dicibile e l'indicibile, poiché la natura intrinseca della realtà è impossibile da cogliere discorsivamente. da qui anche l'interesse verso le metafore di tutto il pensiero cinese. Soltanto il saggio può formarsi un'idea della realtà intrinseca (Dao) ma per comunicarla non può usare il linguaggio discorsivo. L'interesse del pensiero cinese però non è relativo alla metafisica in sè ma a come questa stia in relazione alla molteplicità della realtà, non si perde mai il riferimento all'uomo (cosmologia invece di metafisica).
\section{Il confucianesimo e l'occidente}
La prima rivelazione delle meraviglie dell'oriente è dovuta a Marco Polo che però era un mercante e viaggiatore e non un intellettuale. Il vero incontro culturale si ebbe con i missionari gesuiti Michele Ruggieri (1543-1607) e Matteo Ricci (Macerata 1522- Pechino 1610); in particolare quest'ultimo visse a lungo in Cina e vi morì, comprendendo in profondità questa civiltà. Rendendosi conto che il confucianesimo è rappresentativo dello spirito cinese studiò e tradusse i classici confuciani, rendendosi conto della profonda alterità di quel popolo e la difficoltà nell'intento di assimilarlo al cristianesimo. Finì per adattare il cristianesimo al confucianesimo e soprattutto mostrò le meraviglie delle scienze occidentali per attrarre l'attenzione dei cinesi, facendosi strada fra le elites e convertendo eminenti intellettuali. Dopo poco però gli stessi intellettuali cinesi rilevarono l'incompatibilità di cristianesimo e confucianesimo e mettendo il luce alcune, limitate, modalità di dialogo fra le culture.\\
Durante la dinastia Quing (1644-1911) si riadottò il confucianesimo di epoca Song come cultura ortodossa; si aprì un dibattito contrapponente il confucianesimo metafisico di epoca Ming e Song (influenzato dal taoismo e buddhismo) contrapposto a quello derivante da una più fedele interpretazione delle scritture, più pratico e legato alla vita. Si affermò una linea propendente all'attenzione filologica, e dunque ad un confucianesimo più originale. Durante la dinastia Quing un fatto interessante è il riavvicinamento alla realtà esterna, in particolare società e politica, risvegliandosi da una lunga chiusura introspettiva (causata dall'affermarsi di taoismo e buddhismo), anche a causa dei problemi sorti dal nuovo governo mancese. Ciò risvegliò anche un maggior interesse per le scienze. Mentre la Cina continuava a dibattere sull'eredità confuciana, a partire da metà ottocento, l'interazione con l'occidente diventava sempre più pressante (siamo in pieno colonialismo europeo); in particolare le Guerre dell'Oppio con la Gran Bretagna, che si conclusero con una vittoria occidentale, indussero la Cina ad aprire le frontiere con l'Occidente sia commercialmente che intellettualmente. Gli intellettuali cinesi inizialmente puntarono sul mantenere il confucianesimo ed accorpare i progressi scientifici occidentali, visti come mere applicazioni pratiche. il confronto con l'Occidente portò anche ad una riforma del sistema politico e culturale. Nonostante le contraddizioni e le difficoltà di questo incontro, la visione del mondo occidentale penetrò in profondità in Cina. Nel 1911 si istituì la repubblica e nel 1949 giunse al potere il partito Comunista fondando la Repubblica popolare cinese. Il confucianesimo fu totalmente rigettato dai comunisti, visto come un'ideologia classista e arretrata, in opposizione agli ideali di democrazia e progresso importati dall'occidente. Gli intellettuali cinesi vissero un periodo di declino nella prima parte del Novecento, schiacciati dal pensiero occidentale, ma al contempo si sviluppò un interessante discorso per poter conciliare e confrontare le due visioni del mondo.\\
Quando Mao si impose al potere inizialmente la maggior parte degli intellettuali contribuì al governo, analogamente a quanto gli shi facevano in epoca imperiale, al fine di unificare e potenziare la nazione mediante la coesione dell'ideologia. Se da un lato questo favorì una certa libertà degli intellettuali, d'altra parte portò ad una limitazione dei conservatori, in particolare dei sostenitori del confucianesimo, pensiero che declinò inesorabilmente. In ottica forzatamente marxista-leninista il confucianesimo era la principale causa dell'arretratezza cinese (confucianesimo come "veleno lasciato dal feudalesimo"). Durante la rivoluzione culturale (1966-1976), conclusasi con la morte di Mao, il confucianesimo fu bandito al punto che chi lo avesse professato avrebbe rischiato la vita. Il confucianesimo continuò a vivere fuori dalla Cina, ad Hong Kong e Taiwan. Nel 1958 ad Hong Kong si pubblicò una dichiarazione di intenti dei "nuovi confuciani", aperti al dialogo con l'occidente ma non alla sottomissione politica e culturale, si rendevano conto dei limiti e delle potenzialità del confucianesimo.Un autore confuciano molto ripreso fu Mencio, per la possibilità che offre di integrare valori morali tipicamente occidentali all'interno del confucianesimo. Un tratto interessante del confucianesimo, se comparato alle frange più positiviste della filosofia occidentale, è la presenza di un forte senso della moralità, messa in secondo piano in occidente. \\
A partire dal 1980 il confucianesimo fu diffuso in tutto il mondo. La crisi sorta dalla sfida di accogliere l'Occidente e al contempo di non perdere la propria identità ha favorito un ritorno al confucianesimo, visto come tratto distintivo cinese, fomentando una "febbre del confucianesimo", che fu visto come chiave tutta cinese del superamento del confitto fra capitalismo e marxismo-leninismo, affermando un nuovo modello di vita basato sul concetto di umanesimo confuciano in contrapposizione al materialismo occidentale.

\chapter{Buddhismo}
\section{Siddhartha tra storia e leggenda}
Il buddhismo nasce in India tra V e IV secolo a.C., Buddha ("lo svegliato") è l'appellativo che si diede a Siddhartha Gautama (IV secolo a.C.), figlio di un agiato principe. Alla sua nascita si fa un oroscopo e si scopre che il Siddhartha sarà un grand'uomo, non si sa se politico o religioso. Per scongiurare la via religiosa, ascetica e di rinuncia nell'India del tempo, il bambino è tenuto in una campana di vetro da cui esce a seguito di aver fatto esperienza di 4 eventi: la vecchiaia, la malattia, la morte ed infine un asceta. A seguito di questi eventi non riesce a godere dei piaceri mondani, ha una crisi esistenziale. Ha successo in società e negli studi, si sposa felicemente (importante il fatto che fa esperienza della vita vissuta) ma ad un certo punto abbandona la casa paterna e comincia una vita ascetica (nonostante avesse successo nella vita sociale), resosi conto del fallimento degli insegnamenti tradizionali. Nel momento del congedo concede i suoi beni allo scudiero e va dai maestri ascetici indiani (Yogin) da cui impara la meditazione e ne ha successo, ottenendo poteri straordinari. Nonostante ciò non è appagato dalle brevi condizioni estatiche, vuole conquiste permanenti; prova allora l'ascetismo estremo ma dopo 6 anni si rende conto di esser passato all'eccesso opposto. Allora mangia, beve, ed intraprende la sua via, sta in meditazione sotto un albero senza dissociarsi dalla realtà ma osservandola, scoprendo le verità fondamentali del buddhismo. Fa esperienza del risveglio e della verità che libera, diffonde il suo pensiero (Dharma), fonda una comunità e muore.
 
\section{Il pensiero buddhista}
Le idee fondamentali del buddhismo sono che la vita è sofferenza e questa ha origine nell'attaccamento a ciò che esiste, e dunque al desiderio, poiché in realtà ogni cosa è illusione, anche la percezione di sè è illusione. I due poli dell'esperienza umana sono dunque sete di vita, desiderio e attaccamento da un lato e rifiuto del dolore, repellenza dall'altra. Ciò crea un circolo vizioso che porta al dolore esistenziale; essendo la realtà illusoria per eliminare il dolore bisogna dunque allontanarsi dal mondo/illusione, fino ad arrivare a perdere la propria personalità, seguendo l'ottuplice sentiero.\\
Quanto detto costituisce le \textbf{quattro nobili verità}:
\begin{enumerate}
	\item La vita è sofferenza (più propriamente duhkha: \textbf{insoddisfazione})
	\item La sofferenza ha origine nel \textbf{desiderio} (sete d'esistenza)
	\item La sofferenza può essere eliminata
	\item L'eliminazione della sofferenza è l'\textbf{ottuplice sentiero}
\end{enumerate}
I Buddhisti credono nella rinascita, non è reincarnazione, manca infatti l'idea di anima che entra in un nuovo corpo, è l'energia vitale impersonale che plasma un nuovo individuo. A seconda della propria condotta terrena si può rinascere in forme di vita migliori o peggiori, in ogni caso però la vita è sofferenza quindi la rinascita è in sè negativa perché lega gli esseri al mondo e dunque all'illusione, cioè il dolore. A seconda del karma buono o cattivo accumulato si rinasce in forme di vita superiori o inferiori, il karma non dipende dall'effetto delle azioni ma dall'intensione con cui si fanno. Il ciclo delle rinascite è detto Samsara; la presa di coscienza della illusorietà della realtà, mediante l'ottuplice sentiero, porta al raggiungimento di una dimensione superiore, il Nirvana, in cui ci si svincola dal Samsara. Il Nirvana è irrazionale ed inesprimibile con il linguaggio, di solito ci si riferisce ad esso negativamente. Due figure centrali nella cultura buddhista sono l'arhat, cioè colui che ha raggiunto il Nirvana, e il bodhisattva, che nonostante lo abbia raggiunto decide di reincarnarsi lo stesso per compassione del resto dell'umanità, nella speranza di accompagnarla verso il Nirvana. Gli arhat praticano solo per loro stessi, per raggiungere il Nirvana il prima possibile, a differenza dei bodhisattva.\\
A partire dal  buddhismo di Base, cioè l'insieme delle correnti di pensiero in stretta relazione con la parola di Siddhartha, si sviluppano varie scuole buddhiste che si differenziano sul concetto di liberazione e del migliore percorso da intraprendere per raggiungerla. Le tre più importanti, che ancora oggi sopravvivono sono il Theravada (più tendente al razionalismo), Mahayana e Tantra. La seconda scuola ha particolare importanza nella storia del buddhismo per l'ampia diffusione che ha ottenuto, soprattutto in Cina e Giappone. Il buddhismo Mahayana proclama la superiorità spirituale della via del bodhisattva rispetto a quella dell'arhat in quanto questo si deve fare maestro per gli altri uomini (per compassione) che aspirano a diventarlo. Il tratto fondamentale è che ritiene che tutti possono diventare bodhisattva; è tutt'oggi la scuola più diffusa. Una caratteristica che ha contribuito alla sua diffusione in Cina e Giappone è il suo interesse per la natura umana e la somiglianza con il taoismo. Uno sviluppo interessante del Mahayana è il Tantra (anche detto Vajrayana) incentrato sulla figura del bodhisattva come maestro, caratterizzato dall'esoterismo e dalla pratica di rituali per ottenere l'illuminazione e crede che la conoscenza necessaria per l'illuminazione non è intellettuale. Un'altro sviluppo del Mahayana di particolare interesse è quella devota all'\textbf{Amitabha Buddha}, un bodhisattva che grazie ai suoi meriti può far rinascere nel paradiso occidentale, la "pura terra" dove anche lui vive, dove si può studiare il Dharma sotto la sua guida e quindi diventare bodhisattva e poi Buddha a loro volta, scopo finale di ogni anima nel buddhismo Mahāyāna. Per questo messaggio universalistico questa variante, in forma semplificata, si diffonde molto agevolmente nel popolo.

\section{Buddismo in Cina}
Un tratto del pensiero cinese in contrasto con il buddhismo indiano classico è l'importanza data alla vita vissuta e alla realtà; i cinesi, tentando di eliminare le contraddizioni, formularono una nuova corrente buddhista originale e sofisticata. Il buddhismo in Cina doveva introdursi in una cultura con 1500 anni di storia ed una profonda sedimentazione di pensieri contrastanti con il buddhismo ed in particolare con il confucianesimo. Una prima frizione fra buddhismo e confucianesimo è relativa al rapporto fra ordine monastico e Stato, poiché la concezione peculiare dell'autorità politica era molto diversa in Cina rispetto che nelle terre dove il buddhismo si era originato. In India la comunità monastica (\textbf{sangha}) era avvantaggiata rispetto al re ed era esente da obblighi politici, in quanto considerata fuori dal mondo; alla luce di quanto visto, nel pensiero cinese ciò è inconcepibile, soprattutto se si pensa che l'autorità politica dell'epoca era prevalentemente confuciana. La critica confuciana al buddhismo va però oltre la politica: il confucianesimo è basato sul rispetto della famiglia e sulla pratica di un lavoro utile per la società, mentre lo shanga è totalmente estraneo alla società (si mantenevano con le donazioni), parassitario dal punto di vista confuciano. Per  quanto riguarda i laici, nonostante la popolazione fosse diffidente, il governo era tollerante fintanto che si assolvessero i normali doveri socio-economici e non si incitasse alla sovversione. Si verificarono talvolta persecuzioni su larga scala ma mai paragonabili ai conflitti religiosi europei.\\
La relazione del buddhismo con il taoismo invece, nonostante si vedessero reciprocamente come pensieri rivali, fu più duttile rispetto che quella col confucianesimo e ben presto si verificarono reciproci scambi di idee. Ai livelli alti della società le due restarono distinte ma a livello popolare si mescolarono. Il fatto che il buddhismo si affermò molto rapidamente in Cina denuncia che, nonostante tutto, vi fossero dei punti in comune fra cultura cinese e buddhismo, ne individuiamo alcuni:
\begin{itemize}
	\item L'ideale monastico nonostante fosse percepito come asociale era concorde ad un'antico ideale cinese della "vita ritirata" del saggio.
	\item La meditazione buddhista era affine ad alcune pratiche taoiste di purificazione che, nonostante il diverso fine, costituirono un punto di contatto.
	\item L'idea che il sangha abbia una influenza benefica su chi lo circonda era vicina all'idea del maestro che aiuta i discepoli a trovare il Dao. Inoltre la protezione del monaco era percepita come magica e li si aiutavano anche per superstizione.
\end{itemize}
Il buddhismo ebbe un enorme impatto anche sulla lingua cinese perché per esprimere una visione radicalmente nuova servono nuove parole.
\subsection{Fase embrionale (I-III secolo d.C.)}
Inizialmente gli storici elaborarono questo modello per spiegare la diffusione del buddhismo in Cina: il mondo religioso indiano è molto rigido e diviso in caste, la religione non era aperta alle caste più basse, ed in particolare i mercanti erano mal visti perché spostandosi non riescono a mantenere la loro purezza, inoltre i mercanti agiscono, sono immersi nella realtà da cui la religiosità indiana vuole allontanarsi. Il buddhismo con il suo ideale egualitario accoglie una buona fetta di società indiana. I mercanti si organizzano in comunità che si sostentano con le elemosina delle piccole città di frontiera, una volta che il monastero cresce si divide ed una parte si sposta fondando un altro monastero in città limitrofe, verosimilmente seguendo le vie commerciali ed arrivando al di fuori dell'India fino in Cina.\\
Questo modello porta ad alcune predizioni verificabili archeologicamente secondo cui la diffusione deve essere molto lenta, e dunque il buddhismo deve modificarsi molto nel suo cammino dall'india alla Cina, i traduttori delle prime opere devono essere originari dell'India dunque i loro nomi devono essere indiani. Erik Z\"{u}rcher, grande storico, osserva che questo non è il caso, il buddhismo si diffonde rapidamente, molto simile a quello indiano ed i traduttori erano cinesi di frontiera. Inoltre i centri che avrebbero dovuto dare sostentamento ai monaci erano in realtà tanto piccoli da avere un'economia di sussistenza, senza alcun surplus (fino al III secolo d.C. l'Asia centrale è troppo povera). Ne segue che la diffusione non si ebbe per contatto diretto ma a distanza. Alcune informazioni sul buddhismo del primo secolo sono: 
\begin{itemize}
	\item La presenza di fonti letterarie che attestano cerimonie ibride nell'impero Han in cui si rende omaggio anche ad un Buddha (con possibile presenza di monaci stranieri) 
	\item La presenza di monaci buddhisti con nomi palesemente stranieri che producono le prime traduzioni di testi buddhisti, scritte in cinese pessimo perché tradotto a partire da un interprete che traduce all'impronta le parole di un monaco indiano e che viene trascritto in un cinese con lessico commerciale tradotto a partire da un indiano a sua volta commerciale. Questo attesta il contatto con monaci stranieri indiani.
	\item L'adozione in tombe cinesi di influenze artistiche provenienti dal buddhismo a partire dal II secolo d.C. , tanto simili a quelle indiane da sembrare copiate (contro la tesi per cui ci doveva essere un graduale cambiamento del buddhismo nel suo percorso). Non si può parlare di arte buddhista in Cina, queste immagini non erano comprese ma solamente copiate in funzione apotropaica e per la somiglianza ad altre icone totalmente diverse nelle tombe (orsetti sostituiti da Buddha).
\end{itemize}
L'immagine proposta da Z\"{u}rcher della diffusione del buddhismo è dunque la penetrazione, a partire da città di frontiera, multietniche, di monaci e laici buddhisti.\\
Dal punto di vista del contesto storico la rapida diffusione del buddhismo in Cina (inizialmente come variante esotica del taoismo religioso) è giustificata da un lato dalla caduta dell'impero han e dal conseguente declino della visione del mondo di cui si faceva portavoce, in questo contesto vi era una maggiore apertura verso nuove concezioni. D'altra parte, una tendenza diffusa nei periodi di crisi è quella di focalizzarsi sulla sfera personale e sentimentale, in questo senso il buddhismo offre risposte al dolore e alle incertezze durante il periodo dei tre regni.  Tradizione vuole che il buddhismo entri in Cina a seguito di un sogno dell'imperatore Mingdi in cui vede un'uomo d'oro di grande saggezza, al seguito del quale organizza una spedizione ad ovest da cui i messi tornano con una statua del buddha e altri beni. Ciò porta alla fondazione del monastero del cavallo bianco. \'E un racconto costruito molti secoli dopo dell'arrivo del buddhismo utile ai monaci per trovare consenso nella società.\\
La trasmissione dei testi in epoca Han è saltuaria, proprio a causa dell'occasionalità degli scambi commerciali. Si ha una penetrazione asistematica di molti testi in Cina e inizialmente non si sa come organizzarli, si sviluppa gradualmente un'interesse per la sistematizzazione su base filosofica delle fonti, organizzate in modo da accompagnare il discepolo verso la via dell'illuminazione, da differenze di opinione in questo contesto scaturiranno diverse scuole che privilegiano testi diversi. Altre scuole si ribelleranno a questa tendenza intellettualistica e propenderanno verso una via per l'illuminazione più immediata, che andrà a costituire la scuola Chan del Buddhismo cinese, che è l'unica scuola ufficialmente vigente in Cina. Si formano così due visioni del buddhismo in Cina: una scolastica ed una più irrazionale. Fondamentale è la traduzione del "sutra del loto", facente parte della scuola Mahayana, che riesce a raggiungere fette più ampie di popolazione grazie al suo messaggio universalistico che ammette la possibilità di diventare Buddha a tutti gli esseri viventi. Anche grazie al ricco stile con cui è scritto, questo testo sarà il più diffuso nel buddismo cinese.\\
Il conflitto con la cultura cinese si basa inizialmente su tre punti:
\begin{itemize}
	\item Quella morale è la più sentita e si basa sull'etica della famiglia: il sangha lascia la famiglia, perde il cognome e osserva il celibato (tasto dolente) e ciò nell'ottica cinese tradizionale non adempie al rispetto filiale. Anche l'adozione di abiti stranieri è malvista soprattutto dal popolo. (entrare nel monastero si dice in cinese "uscire dalla famiglia")
	\item Utilitaristicamente il sangha è inutile per la società perché improduttivo e l'inattività porta al crimine
	\item Economicamente inoltre il sangha ha privilegi fiscali e i monasteri tendono ad accumulare ricchezze e nonostante ciò sono istituzioni indipendenti dallo stato e ciò è malvisto. 
	\item Il monastero segue le regole del vinaya, le leggi del monastero, che rendono indipendente il monastero rispetto allo stato, offrendo un tipo di vita alternativo a quello offerto dall'autorità imperiale. Vi è una costante tensione tra mondo monastico ed imperiale. 
\end{itemize}
Si sviluppa dunque una produzione apologetica dei monaci buddhisti.\\
A partire dal III secolo l'Asia Orientale diventa più ricca e le città hanno un surplus tale da poter sostenere monasteri, che proliferano in questo periodo non nelle grandi città perché troppo vicine al Samsara ma neanche troppo lontani da esse per poter ricevere una buona quantità di offerte. 

\subsection{Fase di formazione (III-VI secolo d.C.)}
Nel III secolo si verificano due cambiamenti che avrebbero segnato le sorti del buddhismo: politicamente la caduta dell'impero cinese all'inizio del IV secolo per opera dei popoli turco-mongoli e socialmente l'insediamento del buddhismo nelle fasce alte della società.
\subsubsection{Il Sud}
La caduta dell'impero ebbe come effetto la migrazione di buona parte della popolazione verso sud, dove si riformò un nuovo debole l'impero cinese, due secoli dopo un generale del nord riunificherà l'impero. Nell'impero del sud l'insediamento del buddhismo nelle classi alte continuò e, grazie all'instabilità politica che aveva fatto decadere il confucianesimo, si ebbe un poliedrico fiorire della cultura. In periodi di instabilità il pensiero tende a ripiegarsi sul soggetto e mettere la politica in secondo piano, il taoismo prevalse allora sul buddhismo e si sviluppò un discorso metafisico ed ontologico basato sul commento di Laozi, Zhuangzi e il Libro dei Mutamenti. In questo ambiente il buddhismo mahayana si diffusero, grazie agli insegnamenti dei nuovi dotti sangha, le famiglie potenti finanziano allora la costruzione di monasteri che diventano centri culturali buddhisti. Un fatto interessante è che nel contesto di una società dominata dalle classi il monastero era l'unico modo per un uomo di classe meno abbiente per elevare il suo status sociale (nasce un ordine monastico buddhista femminile che costituisce il 40\% del clero buddhista).   Durante il V secolo il buddhismo si diffonde in tutta la Cina del Sud, degno di nota è il fanatico imperatore buddhista Wu che che proibì il taoismo e celebrò personalmente cerimonie buddhiste. Il famoso maestro Huyiuan fonda un centro buddhista nel 380 d.C. e segna l'inizio di una più profonda comprensione del buddhismo, che si svincola dall'ibridazione col taoismo. Era interessato nella meditazione (yoga) e nell'estasi (dhyana) e mandò un discepolo. Nel V secolo si impone una variante del buddhismo che al posto di vedere il Nirvana come negazione di sè lo vede positivamente come realizzazione della "natura del Buddha", questo concetto diventerà centrale nel buddhismo Chan.
\subsubsection{Il Nord}
La Cina del nord si era unificata sotto la dinastia turco-mongolo che aveva causato la caduta dell'impero e che col tempo si era sinizzata. Sotto le dinastie non cinesi il buddhismo fu favorito da molti sovrani che vedevano i sangha come maghi di corte che assicuravano prosperità e li tenevano in più alta considerazione che i confuciani. Il confucianesimo era una forza sinizzante che i sovrani stranieri controbilanciavano favorendo taoismo e buddhismo, in questo modo il rapporto tra stato e culto era più stretto. Anche a causa dell'influsso culturale dal nord ovest si diffusero buone traduzioni dei testi buddhisti. Il primo a stabilire le basi dottrinali è Dao'an e lui stesso riconosce la somiglianza fra buddhismo e taoismo. Apice della diffusione buddhista nel periodo Toba-Wei. Il benessere materiale condusse però ad un'eccessiva mondanizzazione dei sangha che erano più esposti alle critiche di cui sopra. A fianco dei grandi templi buddhisti istituzionalizzati ne nasceva uno sciame di più piccoli, che erano visti dal potere centrale come minacce per la tendenza messianica (sarebbe arrivato un Buddha salvatore da un momento all'altro) potenzialmente sovversive, che effettivamente si insorsero più volte. \\
La Cina si riunifica nel 589 e la fase di formazione del buddhismo cessa, sia a nord che a sud aveva permeato la società a tutti i livelli.
\subsection{Fase della crescita indipendente (589 d.C.-906 d.C.)}
Dinastie sino-barbare che favoriscono il buddhismo, adottando politiche culturali simili a quella del precedente impero del nord. Essendo il buddhismo largamente diffuso nella popolazione la sua adozione era politicamente favorevole. Lo stato afferma il controllo sui sangha, regolamenta l'elargizione dei titoli (che comportavano grandi vantaggi socio-economici) ed in generale il culto si istituzionalizza, il Canone è stabilito da un ente semi-governativo. Il confucianesimo restò la dottrina ufficiale di stato ma era poco vitale a differenza del buddhismo che vede il proliferare di tante scuole (zong), il termine scuola non è appropriato perché si può far parte di più zong senza contraddizione. Tre principali zong:
\begin{itemize}
	\item Tiantai: fondata dal maestro Zhiyi, basata su una sistematizzazione enciclopedica dei testi che sono visti come propedeutici per avvicinarsi gradualmente al testo principale, contenente la verità: il sutra del loto. Tutti possiedono la natura Buddha, tendenza sincretica fra le varie correnti, una sua versione semplice si diffonde molto nel popolo.
	\item Huayan: crede nell'esistenza di una realtà ultima che si rispecchia nel mondo fenomenico, i due sono legati e interdipendenti. Si basa sul sutra della ghirlanda fiorita.
	\item Chan: scuola di meditazione, ha un'origine leggendaria che risalirebbe al Buddha stesso. Una corrente più scolastica legata alle scritture ed una che crede nell'illuminazione improvvisa ("subitismo") molto popolare. La natura Buddha è in tutti noi e si può realizzare grazie ad un'esperienza improvvisa, inesprimibile. Si oppone al monastero classico e si organizza in piccoli monasteri in cui si lavora e ci si autosostiene, che gli evitava l'accusa di parassitismo. 
\end{itemize}
\newpage
\appendix
\section{Parole}
\begin{itemize}
	\item dao (via)
	\item zhong (mezzo)
	\item li (ordine naturale/senso rituale)
	\item junzi ("uomo di valore")
	\item ren ("concetto di umanità")
	\item shi (letterato-funzionario)
	\item qi (soffio, energia che connette il Cielo all'uomo)
	\item xin (cuore/animo)
	\item xing (natura umana)
	\item Daodejing (Laozi o "Classico della Via e della sua Virtù")
	\item Lao (vecchio)
	\item zi (maestro/bambino)
	\item yi (mutamento)
	\item sangha (monaco buddhista)
	\item dhyana (estasi)
	\item yoga (meditazione)
	\item Dharma (insegnamenti del Buddha)
	\item vinaya (disciplina dei sangha)
	\item zong (nome comune per scuola buddista)
	\item Mahayana (scuola di pensiero buddhista)
	\item arhat (colui che ha raggiunto l'illuminazione)
	\item bodhisattva (persona che pur avendo ormai raggiunto l'illuminazione, e avendo quindi esaurito il ciclo delle sue esistenze terrene, sceglie tuttavia di rinunciare provvisoriamente al Nirvana e di continuare a reincarnarsi, sotto la spinta della compassione, per dedicarsi ad aiutare gli altri esseri umani a raggiungerlo, spendendo per loro i propri meriti)
\end{itemize}
\section{Domande}
\begin{itemize}
	\item Il problema della natura umana nel pensiero cinese
	\item Fisiologia Morale in Mencio
	\item Racconto del cuoco Ding (Zhuangzi)
	\item I riti prima di Confucio
	\item Che parole Confucio fa slittare semanticamente
	\item Pensiero politico confuciano
	\item Il ruolo dei Classici in Confucio
	\item Le tre metafore taoiste
	\item Lo xin e xing in Mencio
	\item Spontaneità in Confucianesimo e Buddhismo
	\item Diffusione del Buddhismo per Zurcher
\end{itemize}
\end{document}